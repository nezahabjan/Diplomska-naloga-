\documentclass[12pt,a4paper]{amsart}
% ukazi za delo s slovenscino -- izberi kodiranje, ki ti ustreza
\usepackage[slovene]{babel}
%\usepackage[cp1250]{inputenc}
%\usepackage[T1]{fontenc}
\usepackage[utf8]{inputenc}
\usepackage{amsmath,amssymb,amsfonts}
\usepackage{url}
%\usepackage[normalem]{ulem}
\usepackage[dvipsnames,usenames]{color}


% ne spreminjaj podatkov, ki vplivajo na obliko strani
\textwidth 15cm
\textheight 24cm
\oddsidemargin.5cm
\evensidemargin.5cm
\topmargin-5mm
\addtolength{\footskip}{10pt}
\pagestyle{plain}
\overfullrule=15pt % oznaci predolgo vrstico


% ukazi za matematicna okolja
\theoremstyle{definition} % tekst napisan pokoncno
\newtheorem{definicija}{Definicija}[section]
\newtheorem{primer}[definicija]{Primer}
\newtheorem{opomba}[definicija]{Opomba}

\renewcommand\endprimer{\hfill$\diamondsuit$}


\theoremstyle{plain} % tekst napisan posevno
\newtheorem{lema}[definicija]{Lema}
\newtheorem{izrek}[definicija]{Izrek}
\newtheorem{trditev}[definicija]{Trditev}
\newtheorem{posledica}[definicija]{Posledica}


% za stevilske mnozice uporabi naslednje simbole
\newcommand{\R}{\mathbb R}
\newcommand{\N}{\mathbb N}
\newcommand{\Z}{\mathbb Z}
\newcommand{\C}{\mathbb C}
\newcommand{\Q}{\mathbb Q}


% ukaz za slovarsko geslo
\newlength{\odstavek}
\setlength{\odstavek}{\parindent}
\newcommand{\geslo}[2]{\noindent\textbf{#1}\hspace*{3mm}\hangindent=\parindent\hangafter=1 #2}


% naslednje ukaze ustrezno popravi
\newcommand{\program}{Finančna matematika} % ime studijskega programa: Matematika/Finan"cna matematika
\newcommand{\imeavtorja}{Neža Habjan} % ime avtorja
\newcommand{\imementorja}{prof.~doc.~dr. Matjaž Črnigoj} % akademski naziv in ime mentorja
\newcommand{\naslovdela}{Posebnosti pri vrednotenju podjetja z internim trgom delnic - podjetje X}
\newcommand{\letnica}{2019} %letnica diplome


% vstavi svoje definicije ...




\begin{document}

% od tod do povzetka ne spreminjaj nicesar
\thispagestyle{empty}
\noindent{\large
UNIVERZA V LJUBLJANI\\[1mm]
FAKULTETA ZA MATEMATIKO IN FIZIKO\\[5mm]
\program\ -- 1.~stopnja}
\vfill

\begin{center}{\large
\imeavtorja\\[2mm]
{\bf \naslovdela}\\[10mm]
Delo diplomskega seminarja\\[1cm]
Mentor: \imementorja}
\end{center}
\vfill

\noindent{\large
Ljubljana, \letnica}
\pagebreak

\thispagestyle{empty}
\tableofcontents
\pagebreak

\thispagestyle{empty}
\begin{center}
{\bf \naslovdela}\\[3mm]
{\sc Povzetek}
\end{center}
% tekst povzetka v slovenscini
V povzetku na kratko opi"si vsebinske rezultate dela. Sem ne sodi razlaga organizacije dela -- v katerem poglavju/razdelku je kaj, pa"c pa le opis vsebine.
\vfill
\begin{center}
{\bf Angle"ski naslov dela}\\[3mm] % prevod slovenskega naslova dela
{\sc Abstract}
\end{center}
% tekst povzetka v anglescini
Prevod zgornjega povzetka v angle"s"cino.

\vfill\noindent
{\bf Math. Subj. Class. (2010):} navedi vsaj eno klasifikacijsko oznako -- dostopne so na \url{www.ams.org/mathscinet/msc/msc2010.html}  \\[1mm]
{\bf Klju"cne besede:} navedi nekaj klju"cnih pojmov, ki nastopajo v delu  \\[1mm]
{\bf Keywords:} angle"ski prevod klju"cnih besed
\pagebreak



% tu se zacne besedilo seminarja
\section{\textbf{UVOD}}
Podjetja po svetu dandanes delujejo v mnogih različnih oblikah, znotraj teh pa se pojavljajo razlike še glede na organiziranost, vodenje, smernice delovanja,... Vse to ustvarja veliko heterogenost v podjetniškem prostoru. Seveda pa je, če želimo, da podjetje posluje uspešno in z dobičkom, potrebno vse te dejavnike izbirati skrbno in premišljeno. Vsaka vrsta organiziranosti namreč ne ustreza vsem trgom, delavcem, panogam,... \\
V slovenskem prostoru se tako nahaja podjetje, ki je po svoji organiziranosti unikat. Kljub temu, da je precej zaprto, kar se tiče vlaganja tretjih oseb v njegove delnice, posluje z odliko in si v Evropi lasti 60\% tržni delež na področju sesalnih enot, ki so njegov glavni proizvod. To je podjetje X, ki je delniška družba, s svojimi delnicami pa upravlja na zaprtem, internem trgu. Neizpostavljenost širšemu trgu ali borzi zato povzroča njihovo precej nizko vrednost, še posebej, ker podjetje uživa izjemno dobro ime v javnosti, kar bi za vlagatelje lahko pomenilo velik interes.\\
V svojem diplomskem delu bi zato rada analizirala in ocenila padec vrednosti delnice podjetja X, zaradi njegove zaprtosti. Pri tem bom preučila obnašanje delničarjev podjetja, torej če in kdaj imajo interes povečati oziroma zmanjšati svojo naložbo vanj. Ob vrednotenju bom kot glavnega morala oceniti diskontni faktor, s katerim bom prihodnje denarne tokove prenesla na sedanjo vrednost. Pri tem si bom pomagala z dvema pomembnejšema konceptoma odbitkov, ki sta zaradi zaprtosti podjetja ključnega pomena, in sicer s tržljivostjo in obvladljivostjo.
\newpage





%%%%%%%%%%%%%%%%%%%%%%%%%%%%%%%%%%


\section{\textbf{TEORIJA OCENJEVANJA VREDNOSTI}}

Pri ocenjevanju vrednosti poznamo tri glavne načine vrednotenja, od katerih vsi temeljijo na ekonomskih načelih ravnovesja cen, pričakovanih koristi ali substituciji. To so način tržnih primerjav, na sredstvih zasnovan način in na donosu zasnovan način, vsak od njih pa vključuje podrobnejše metode uporabe. Med temi moramo izbirati previdno in v obzir vzeti vrsto sredstva, ki ga ocenjujemo, razpoložljive informacije, ki jih imamo, naš namen ocenjevanja ter ustrezne prednosti in slabosti posamezne metode. 


\subsection{NAČIN TRŽNIH PRIMERJAV}:\\

Ta način temelji na predpostavki, da dajejo kupoprodaje podobnih premoženj (nujno med nepovezanimi osebami), kot je naše ocenjevano, dovolj zgovorne dokaze o vrednosti ocenjevanega premoženja. 
Vrednost ocenjevanega sredstva tako dobimo preko primerjave z vrednostmi podobnih sredstev, ki so že ovrednotena, torej so informacije o cenah že na voljo. Uporaba pride v poštev predvsem v primerih, ko je bilo ocenjevano sredstvo nedavno prodano v poslu, ki je primeren za proučevanje, ko se sredstvo dejavno javno trži, ali pa s podobnimi sredstvi obstajajo nedavni posli. Kadar se primerljiva tržna informacija ne nanaša na točno ali v bistvu enako sredstvo, mora ocenjevalec vrednosti opraviti primerjalno analizo tako podobnosti kot tudi razlik po kakovosti in količini med primerljivimi sredstvi in ocenjevanim sredstvom. (MSOV 2017) \\

Metode načina tržnih primerjav so :
\begin{itemize}
\item metoda primerljivih kupoprodaj podjetij
\item metoda primerljivih podjetij, uvrščenih na borzi
\end{itemize}

\subsubsection{METODA PRIMERLJIVIH PODJETIJ, UVRŠČENIH NA BORZI}:\\

Pri ocenjevanju podjetij je zlasti pomembno, da ocenjevalec za primerjavo izbere tista podjetja, ki imajo enake dejavnike tveganja (okoliščine ponudbe in povpraševanja, prodajne poti, način dobave,..) kot ocenjevano. Poleg tega moramo upoštevati tudi podobnosti na področju trga, proizvodov in storitev, velikosti podjetja, preteklih podatkov iz poslovanja, območja delovanja (geografsko),... Določiti mora tudi primerno število podjetij za primerjavo, na kar vpliva predvsem stopnja primerljivosti. Najboljši vzorec za primerjavo naj bi obsegal od pet do sedem podjetij. (Praznik, 2004, 85.)\\
Izračun vrednosti preko metode primerljivih podjetij poteka po konceptu mnogokratnikov ocenjevanja vrednosti. Pri tem pomnožimo temeljne finančne spremenljivke podjetja (dobiček pred davkom, čisti dobiček, obseg prodaje, bruto denarni tok,...) z mnogokratnikom, izraženim kot:
\begin{equation}
MV=\frac{1}{c}=\frac{1}{d-g}
\end{equation}
Kjer je:\\
$MV$... mnogokratnik ocenjevanja vrednosti\\
$c$... mera uglavničenja (kapitalizacije)\\
$d$... diskontna mera\\
$g$... stopnja rasti\\
Mnogokratnik dobimo preko borznih kotacij primerljivih podjetij, pri čemer moramo vzeti tisto tržno ceno delnice, ki je aktualna na datum ocenjevanja vrednosti. Prav tako  pa moramo za izračun vseh ostalih vrednosti preko mnogokratnikov, jemati podatke istega dne oziroma iz istega časovnega obdobja, ne glede na datum vrednotenja. Časovno obdobje je lahko obdobje zadnjih 12 mesecev, zadnje obračunsko leto, povprečje zadnjih 3 let,...
\begin{itemize}
\item \textbf{Mnogokratniki lastniškega kapitala}:\\
Preko teh mnogokratnikov dobimo vrednost lastniškega kapitala, pri čemer so njihove primernosti uporabe različne. 
\begin{itemize}
\item $cena/cisti\ dobicek\ na\ delnico$ - primerljiva podjetja imajo enako strukturo celotnega kapitala, kot naše ocenjevano podjetje.
\item $cena/(cisti\ dobicek\ +\ nedenarni\ stroski\ na\ delnico)$ - amortizacija predstavlja velik del stroškov.
\item $cena/dobicek\ pred\ davkom$ - ocenjevano podjetje ima drugačno davčno stopnjo od primerljivih.
\item $cena/prihodki\ iz\ prodaje\ na\ delnico$ - upoštevamo stalnost kupcev in podobno operativno delovanje med podjetji.
\item $cena/knjigovodska\ vrednost$ - uporabimo, ko knjigovodske vrednosti temeljijo na tržnih cenah.
\end{itemize}
\item \textbf{Mnogokratniki celotnega kapitala}:\\
Uporabljamo jih, kadar se podjetja od ocenjevanega razlikujejo po sestavi sredstev in kapitala, z njimi pa dobimo vrednost celotnega kapitala. Ne glede na stopnjo obvladovanja ocenjevanega deleža, so navadno pogosteje uporabljeni kot mnogokratniki lastniškega kapitala.
\begin{itemize}
\item $cena\ celotnega\ kapitala\ (MVIC)/dobicek\ pred\ stroski\ financiranja\ (EBITDA)$
\item $cena\ celotnega\ kapitala\ (MVIC)/dobicek\ pred\ stroski\ financiranja\ in\ davki\ (EBIT)$
\item $cena\ celotnega\ kapitala\ (MVIC)/knjigovodska\ vrednost\ celotnega\ kapitala$
\end{itemize}
\end{itemize}
Med vsemi podanimi mnogokratniki vedno izberemo tiste, za katere imamo zanesljive in ustrezne podatke ter nam najbolje prikazujejo pričakovano rast podjetja in pa dejavnike tveganja. Iz nabora vseh poračunanih vrednosti za primerljiva podjetja pa nato dejansko vrednost mnogokratnika lahko določimo kot aritmetično povprečje, mediano, spodnji/zgornji kvartal,... \\
Ko pridobimo vrednost ocenjevanega deleža, je le to potrebno prilagoditi glede na stopnjo tržljivosti in pa obvladovanja, o čemer pa več sledi v kasnejših poglavjih.

\subsubsection{METODA PRIMERLJIVIH KUPOPRODAJ PODJETIJ}:\\

Pri tej metodi vrednost ocenjevanega deleža določamo preko primerljivih kupoprodaj podjetij, pri čemer moramo biti pozorni, da so se izbrani posli res izvajali med nepovezanimi osebami. Navadno obravnavamo nakup ali prodajo obvladujočega deleža, število primerljivih kupoprodaj pa je zaradi manjšega števila razpoložljivih kupoprodaj med neodvisnimi strankami, manjše kot pri izbiri primerljivih podjetij.\\
Koncept vrednotenja po tej metodi zopet poteka preko mnogokratnikov, popolnoma na enak način kot pri prvi. Dodatno moramo le še ugotoviti kaj točno je vključeno v ceno prodaje in mogoče ni predmet našega ocenjevanja vrednosti ter posledično izvesti ustrezne prilagoditve. Prav tako se tudi pri metodi primerljivih kupoprodaj pojavita odbitka za pomanjkanje obvladljivosti in tržljivosti, vendar moramo biti pozorni na možnost, da cene odražajo elemente vrednosti za naložbenika in ne poštenih tržnih vrednosti, kar lahko povzroči sporne prilagoditve. (Praznik, 2004)


\subsection{NA SREDSTVIH TEMELJEČ NAČIN}:\\

Ta način se sicer v splošnem redkeje uporablja pri vrednotenju podjetij, v poštev pa pride v primerih, ko je podjetje še v zgodnji fazi ali gre za zagonsko podjetje, kjer dobičkov in/ali denarnega toka ni mogoče zanesljivo določiti in so primerjave z drugimi podjetji po načinu tržnih primerjav praktično nemogoče ali nezanesljive. (MSOV, 2017)\\
Metode na sredstvih temelječega načina so :
\begin{itemize}
\item metoda prilagojenih knjigovodskih vrednosti
\item metoda presežnih donosov
\end{itemize}

\subsubsection{METODA PRILAGOJENIH KNJIGOVODSKIH VREDNOSTI}:\\

S to metodo prilagodimo vse obveznosti in vsa sredstva podjetja (materialna in nematerialna) na tržno vrednost, ne glede na to, ali so vključena v bilanco stanja ali ne. Vrednost lastniškega kapitala je posledično razlika med tako ocenjenimi vrednostmi sredstev in obveznosti.\\
Tudi pri tej metodi moramo na koncu uporabiti prilagoditve vrednosti, in sicer vrednosti sredstev. Te lahko temeljijo na predpostavki poslujočega podjetja ali pa likvidacije le tega. Razlika je namreč ta, da pri ocenjevanju vrednosti sredstev poslujočega podjetja izhajamo iz njihove najgospodarnejše uporabe.\\
V splošnem vrednotenje sredstev podjetja delimo na vrednotenje nepremičnin in pa strojev in opreme. Vrednotenje nepremičnin poteka na tri načine:
\begin{itemize}
\item \underline{Nabavnovrednostni način}\\
Uporabimo metodo za amortizacijo zmanjšane reprodukcijske vrednosti, kjer v osnovi upoštevamo ocenjeno nabavno vrednost za nadomestitev vrednotenega objekta, z novim, točno enakim. Le to nato postopoma zmanjšujemo glede na funkcionalno zastaranje, fizično dotrajanost, ekonomsko in popravljivo funkcionalno zastaranost in pridemo do tržne vrednosti objekta. Prišteti mu moramo še vrednost zemljišča in tako dobimo tržno vrednost celotne nepremičnine.
\item \underline{ Na donosu zasnovan način}\\
V uporabi sta metodi uglavničenja in diskontiranja denarnih tokov. Pri prvi moramo tako opredeliti ekonomski donos nepremičnine (prihodek iz najema, ki ga zmanjšamo za stroške vzdrževanja objekta), pred obdavčenjem. Pri metodi diskontiranja denarnih tokov pa po opredelitvi obdobja eksplicitne napovedi čistih dobičkov od sredstva pred davkom, določimo še preostalo vrednost kot iztržljivo vrednost sredstva po obdobju napovedi. Zneske diskontiramo in njihov seštevek je današnja vrednost nepremičnine.
\item \underline{Način tržnih primerjav}\\
Pri tem načinu primerjamo prodaje podobnih objektov s podobnimi značilnostmi kraja nahajanja, ki so se izvajale v čim bližnji preteklosti. V relavanten izbor kupoprodaje vključimo glede na časovni odmik, na datum ocenjevanja, razmerje med zemljiščem in stavbo, komunalno opremljenost, lokacijo, starost in fizično stanje nepremičnine, dostop, vidike okolja in financiranja,... 
\end{itemize}
Pri vrednotenju strojev in opreme sicer lahko uporabljamo vse tri omenjene metode vrednotenja nepremičnin, a se zaradi manjše ustreznosti ostalih dveh pogosteje uporablja le nabavnovrednostni način.\\
Na področju prilagoditev vrednosti tokrat uporabljamo le odbitek za neobvladovanje, ker se ocenjena vrednost običajno nanaša za obvladujoči delež. Odbitek za pomanjkanje tržljivosti pa je pri tej metodi lahko sporen, saj je dejansko tržljivost obvladujočega deleža težko določiti, čeprav nekateri menijo, da je polno prodajljiv.

\subsubsection{METODA PRESEŽNIH DONOSOV}:\\

Ta metoda se uporablja le v primerih, ko druge ni mogoče uporabiti. Tokrat ocenjujemo tržno vrednost opredmetenih osnovnih sredstev in presežnih donosov (ocenjena raven dobička, zmanjšana za delež zahtevane mere donosa opredmetenih sredstev). Upoštevamo tudi obe že prej omenjeni prilagoditvi v odvisnosti od tega, kako velik delež ocenjujemo. 


\subsection{NA DONOSU ZASNOVAN NAČIN}:\\

``Strošek kapitala je pričakovana mera donosa, ki jo zahteva trg za vlaganja v določeno naložbo." (Praznik, 2004, 49.) Je torej bistvo pri ocenjevanju vrednosti podjetij,  še posebej pri ``na donosu zasnovanem načinu". Sedanjo vrednost ocenjevanega deleža dobimo namreč tako, da prihodnje denarne tokove diskontiramo s stroškom kapitala. Ta strošek je določen s trgom, saj predstavlja mero donosa, ki jo vlagatelj želi, upoštevajoč donose vseh ostalih naložb na trgu. Pri tem v obzir v največji meri vzame njeno tveganost, torej verjetnost, da bo pričakovani donos res uresničen. Sestavine celotnega kapitala se po svojih stroških razlikujejo, kar nam meri WACC (skupno tehtano povprečje stroškov celotnega kapitala), ki zajema tudi delež posamezne vrste kapitala v celotnem. Strošek kapitala torej ni odvisen od naložbenika ampak od naložbe, in sicer temelji na pričakovanih donosih v tržnih in ne knjigovodskih vrednostih. Odraža tri glavne sestavine: realno mero donosa, ki jo naložbenik pričakuje v zameno za vložek (predpostavljamo netveganost), pričakovano inflacijo in pa tveganje v zvezi s časom prejema denarnega toka.\\
Kot omenjeno, je celoten kapital sestavljen iz več vrst, in sicer v grobem iz dolžniškega in lastniškega kapitala. Strošek dolga je zato enak tržnim obrestnim meram za posojila, zmanjšanim za davek. Na drugi strani je strošek lastniškega kapitala znesek izplačanih dividend podjetja delničarjem, ki pa nima davčnega ščita.\par
``Tehtano povprečje stroškov celotnega kapitala torej določimo kot:

\begin{equation}
WACC=(k_e * W_e)+(k_p*W_p)+(k_{d(pt)}*(1-t)*W_d)
\end{equation}

Kjer so:\\
$k_e$... strošek kapitala, ki velja za navadne delnice\\
$W_e$... odstotek kapitala, ki velja za navadne delnice v sestavi celotnega kapitala, opredeljeni na osnovi tržnih vrednosti\\
$k_p$... strošek kapitala, ki velja za prednostne delnice\\
$W_p$... odstotek kapitala, ki velja za prednostne delnice v sestavi celotnega kapitala, opredeljeni na osnovi tržnih vrednosti\\
$k_{d(pt)}$... strošek dolga pred obdavčitvijo\\
$t$... davčna stopnja\\
$W_d$... odstotek dolga v sestavi celotnega kapitala, opredeljeni na osnovi tržnih vrednosti"\\
(Praznik, 2004, 52.)\\

Če se podrobneje osredotočimo na strošek lastniškega kapitala lahko povemo, da ta vključuje netvegano mero donosa (običajno mera donosa državnega vrednostnega papirja - obveznice, zakladnice,... -, ki je manj tvegan) in pribitek za kapitalsko tveganje (sestavljen iz splošnega pribitka za kapitalsko tveganje glede na ostale naložbe na trgu in pribitkov za majhnost podjetja ter posebnih tveganj). Mero posebnega tveganja, označeno z $\beta$, pa pridobimo iz mer tveganja podjetij, ki so prisotna v isti panogi kot ocenjevano podjetje ter kotirajo na borzi.\\
``Po modelu ocenjevanja dolgoročnih sredstev (CAPM) tako strošek kapitala izračunamo kot:

\begin{equation}
E(R_d)=R_n+\beta*ERP
\end{equation}
Kjer so:\\
$R_n$... mera donosa netveganega vrednostnega papirja\\
$\beta$... koeficient beta, mera posebnega tveganja\\
$ERP$... pribitek za kapitalsko tveganje (če je $\beta=1$)"\\
(Praznik, 2004, 54.)\\
 Model CAPM temelji na razumnosti naložbenikov in njihovi nenaklonjenosti k tveganju, učinkoviti razpršenosti portfeljev racionalnih vlagateljev, transparentnosti in likvidnosti trga,... Obstajajo pa tudi drugi modeli za določanje stroškov lastniškega kapitala:
\begin{itemize}
\item \underline{Model dograjevanja}\\
Ta za razliko od CAPM ne vsebuje koeficienta $\beta$, zato je postavke, ki jih ta odraža potrebno še dodatno upoštevati.
\item \underline{Model diskontiranega denarnega toka}\\
Višino diskontne mere pri tem modelu dobimo iz diskontiranja prihodnjih izplačil dividend, na današnjo vrednost, ki predstavlja današnjo tržno vrednost delnice:
\begin{equation}
PV=\frac{NCF0*(1+g)}{d-g}
\end{equation}
Kjer je:\\
$PV$... sedanja tržna vrednost delnice\\
$NCF0$... dividenda, izplačana na eno delnico neposredno pred časom vrednotenja\\
$g$... predpostavljena konstantna stopnja rasti\\
$d$... diskontna mera, ki jo iščemo\\
\\
\item \underline{Model uravnoteženega določanja stroška kapitala}\\
Pri tem modelu se koeficient $\beta$ razdeli na posamezne koeficiente, od katerih vsak odraža občutljivost lastniškega kapitala posameznega od dejavnikov tveganja, v primerjavi z občutljivostjo trga.
\end{itemize}
Izmed vseh naštetih je metoda CAPM daleč najpogosteje uporabljena predvsem zaradi velike dostopnosti empiričnih podatkov, ki jih za izračun po tej metodi potrebujemo.


Način vrednotenja, ki ga trenutno obravnavamo (na donosu zasnovan), obsega dve metodi:
\begin{itemize}
\item metoda diskontiranega denarnega toka
\item metoda uglavničenja
\end{itemize}


\subsubsection{METODA DISKONTIRANEGA DENARNEGA TOKA}:\\

"Uporabljena metoda temelji na predpostavki, da je vrednost naložbe enaka vsoti vseh prihodnjih donosov, ki jih naložba zagotavlja lastniku. Pri čemer se vsak donos diskontira na sedanjo vrednost z diskontno mero, ki izraža časovno vrednost denarja in mero tveganja, povezano z možnostjo uresničitve pričakovanega donosa." (Praznik, 2004, 69.)\\
Prednost metode diskontiranja je predvsem teoretična korektnost (zajema sedanjo vrednost vseh v prihodnosti uresničenih denarnih tokov), naročniki pa jo ob vedno bolj množični uporabi sprejemajo kar kot temeljno metodo pri ocenjevanju vrednosti. Po drugi strani pa je projekcija prihodnjih donosov lahko zahtevna, prav tako pa tudi določitev ustrezne diskontne mere. Poleg tega nam težave lahko povzročata tudi odločanje o ustreznem številu let napovedi, ki bo najbolj merodajno in določanje pogostosti predvidevanega donosa.\\
Pri metodi diskontiranja za lastniški kapital morajo donosi, ki jih uporabljamo za prevrednotenje obsegati le znesek, namenjen lastnikom navadnih delnic, diskontna mera pa le strošek lastniškega kapitala. V primeru računanja vrednosti za celoten kapital pa vzamemo za donose zneske, razpoložljive za vse naložbenike v podjetje, tudi lastnike prednostnih delnic in posojilodajalce, za diskontno mero pa vzamemo prej opisani WACC. \\

Denarni tok podjetja, s katerim torej določamo njegovo vrednost, imenujemo prosti denarni tok (FCF - Free Cash Flow). Ta je najprimernejše merilo za ekonomski donos, saj je to znesek, s katerim podjetje lahko razpolaga, ne da bi ogrozilo načrtovano dejavnost  in denarni tok v prihodnosti.\\
\\
``Za lastniški kapital ga opredelimo takole:\\
\textbf{čisti dobiček} (po davku na dobiček)\\
\textbf{+nedenarni stroški} (amortizacija, povečanja dolgoročnih rezervacij)\\
\textbf{-naložbe v osnovna sredstva }(v obsegu, ki je potreben za uresničitev načrtovanega obsega aktivnosti)\\
\textbf{+denar, pridobljen z dezinvestiranjem} (v obsegu, ki je potreben za prilagoditev sredstev načrtovanemu obsegu aktivnosti)\\
\textbf{-vlaganja v obratni kapital }(obsegu, ki je potreben za uresničitev načrtovanega obsega aktivnosti)\\
\textbf{+denar iz novo najetih posojil} (v obsegu, ki je potreben za uresničitev načrtovanega obsega aktivnosti)\\
\textbf{-odplačila anuitet iz najetih posojil}\\
\textbf{=čisti denarni tok  za lastniški kapital}\\ " (Praznik, 2004, 70.)\\

Če namesto čistega dobička upoštevamo dobiček iz poslovanja, prilagojenega za davčno stopnjo na dobiček in odvzamemo zadnji dve postavki (denar in novo najetih posojil in pa odplačila anuitet) pa dobimo FCF za celoten kapital podjetja.\\

Sledi diskontiranje na današnjo vrednost, za kar uporabimo osnovni princip financ:
\begin{equation}
PV=\frac{CF_1}{1+r}+\frac{CF_2}{(1+r)^2}+\frac{CF_3}{(1+r)^3}+\ldots+\frac{CF_n}{(1+r)^n}
\end{equation}
Kjer so:\\
$PV$...sedanja vrednost\\
$CF1, CF2, CF3$,... ...denarni tokovi v prihodnosti\\
$r$...zahtevana stopnja donosa\\
$1,2,3,...n$ ...število časovnih obdobij diskontiranja\\

Vrednotenje na osnovi diskontiranja denarnih tokov v večini primerov predpostavlja trajni obstoj podjetja, torej sredstva nimajo določene življenjske dobe, zato je denarne tokove potrebno napovedati zelo daleč v prihodnost. Posledično vrednotenje poenostavimo tako, da prihodnost razdelimo na obdobje eksplicitne napovedi denarnih tokov in na preostalo obdobje, od konca eksplicitnega napovedovanja do neskončnosti. \\
Sedanje vrednosti pričakovanih donosov torej preračunamo z diskontiranjem napovedanih tokov, za določanje preostale vrednosti pa lahko uporabimo več metod, med katerimi je najpogostejša in najustreznejša metoda Gordonovega modela rasti. Pri tem moramo poleg že določenih zneskov v prihodnosti in izbranega obdobja eksplicitne napovedi le teh, oceniti še dolgoročno konstantno stopnjo rasti denarnega toka, po obdobju konkretne napovedi. V tem primeru se zgornja enačba prepiše v:

\begin{equation}
PV=\frac{CF_1}{1+r}+\frac{CF_2}{(1+r)^2}+\frac{CF_3}{(1+r)^3}+\ldots+\frac{CF_n}{(1+r)^n}+\frac{\frac{CF_n*(1+g)}{r-g}}{(1+r)^n}
\end{equation}
Kjer nova oznaka $g$ pomeni ocenjeno konstantno stopnjo rasti denarnega toka v prihodnosti.\\
\\
Vrednosti, dobljene po metodi diskontiranja prilagodimo z uporabo prej omenjenih odbitkov za pomanjkanje tržljivosti in neobvladovanje, kar je odvisno od velikosti deleža, ki ga ocenjujemo.


\subsubsection{METODA UGLAVNIČENJA (KAPITALIZACIJE)}:\\

Ta metoda je poenostavljena različica prejšnje, saj ta ne temelji na natančni projekciji prihodnjih donosov, pač pa predpostavlja konstantno rast ali padec normaliziranega letnega donosa, v neskončnost. Kot prvo, tudi metodo uglavničenja lahko uporabljamo za obe vrsti kapitala (lastniški ali celotni). Pri prvi zopet za donos uporabimo vsa razpoložljiva sredstva za izplačilo dividend navadnim delničarjem, pri drugi pa vsem možnim naložbenikom. Mera uglavničenja pri lastniškem kapitalu je tako strošek lastniškega kapitala, pri celotnem pa ponovno WACC.\\
Pri tej metodi spremembe v pričakovanih prihodnjih donosih izražamo preko mere uglavničenja, ne pa preko posebnih napovedi donosov, kot pri prejšnjem primeru. Velja pa, da ob upoštevanju konstantne stopnje rasti donosa pri opredelitvi napovedi donosov v prejšnjem modelu, dasta obe metodi enak rezultat.\\
Spremembe donosov pri metodi uglavničenja torej dosežemo preko stopnje rasti iz že omenjenega Gordonovega modela:

\begin{equation}
c=d-g
\end{equation}
Kjer je:\\
$c$... mera uglavničenja\\
$d$... diskontna mera\\
$g$... stopnja rasti (teoretično v neskončnosti)\\
(Praznik, 2004)\par

V primeru upoštevanja WACC kot diskontne mere torej dobimo mero uglavničenja za celoten kapital ($c_k$):
\begin{equation}
c_k=WACC-g
\end{equation}

Osnovna formula, po kateri torej poračunamo vrednost deleža po metodi uglavničenja, je oblike:
\begin{equation}
PV=\frac{NDT_1}{c}
\end{equation}
Kjer je:\\
$PV$... sedanja vrednost deleža\\
$NDT_1$... čisti denarni tok, pričakovan takoj po dnevu ocenjevanja vrednosti\\
$c$... mera uglavničenja (v tem primeru enaka diskontni meri)\\
(Praznik, 2004)\par

Ob dodatnem upoštevanju rasti donosa ($g$), pa se ta preoblikuje v:
\begin{equation}
PV=\frac{NDT_0*(1+g)}{d-g}
\end{equation}
Kjer moramo kot čisti denarni tok vzeti tistega, neposredno pred datumom ocenjevanja, ker pričakujemo njegovo rast/padec v prihodnosti, in sicer že takoj v obdobju $1$.\par

Kot vidimo, se v vsakem izmed načinov vrednotenja pojavi potreba po prilagoditvah dobljene vrednosti. Te se predvsem nanašajo na velikost lastniškega deleža, ki ga ocenjujemo. Ker bo to eden ključnih konceptov tudi pri cenitvi deleža podjetja Domel, se osredotočimo v nadaljevanju na omenjene prilagoditve.




\section{\textbf{PRILAGODITVE VREDNOSTI, DOBLJENE PRI VREDNOTENJU}}

Vrednotenje podjetja se večkrat izvaja na podlagi predpostavk, ki pa v končnem vrnejo le približno vrednost in ne točne. Deleži, ki jih ocenjujemo se lahko namreč razlikujejo glede na velikost, lastninske pravice, posedovane poslovne informacije deležnika,... Prav tako moramo v obzir vzeti stanje panoge, s katero se podjetje ukvarja ter ekonomije, strukturo kapitala podjetja, njegovo odprtost za investitorje,... Prilagoditve vrednosti se tedaj v računanje sedanje prave vrednosti podjetja vnese preko t.i. odbitkov in pribitkov. Z vidika lastniškega deleža poznamo dva glavna koncepta le teh, in sicer odbitke in pribitke, ki so povezani z bistvom ocenjevanega podjetja in tiste, povezane z značilnostmi lastništva v njem. Prva skupina je neodvisna od velikosti deleža lastnika in njegovih upravljalskih zmožnosti. Upoštevamo jih že pred odbitki iz druge skupine, in sicer jih dodajamo že diskontni meri, meri uglavničenja ali pa ustrezno zmanjšamo mnogokratnik (odvisno od vrste metode, ki jo uporabljamo). Primeri takšnih odbitkov so: odbitek zaradi vpliva ključne osebe, zaradi sodnih sporov, tržnih tveganj, ekoloških problemov,... \\
Pomembnejša pa je druga skupina odbitkov, ki se deli na dva večja razreda, in sicer na odbitke zaradi neobvladljivosti in zaradi pomanjkanja tržljivosti. Te pa, v nasprotju s prvo skupino, navadno upoštevamo šele na koncu, ko smo sredstvo že ovrednotili. Dobljena ocena vrednosti nam torej predstavlja osnovo, na katero apliciramo prilagoditev in je lahko različnih zvrsti. Predstavlja lahko vrednost za obvladujočega lastnika, za manjšinskega lastnika v lastniško odprtem ali zaprtem podjetju, strateško vrednost za naložbenika,... (Praznik, 2004, 103.)\par
Dobljene osnovne vrednosti pa se med seboj razlikujejo tudi zaradi različnih metod, po katerih ocenjujemo. Tako pri na donosu zasnovanem načinu dobimo vrednost obvladujočega ali manjšinskega lastnika (odvisno od vrste uporabljenih denarnih tokov), kjer pa je privzeta popolna tržljivost. Pri uporabi načina tržnih primerjav dobimo pri metodi primerljivih podjetij, uvrščenih na borzi, vrednost za manjšinskega lastnika, zopet ob predpostavki popolne tržljivosti. Pri metodi primerljivih kupoprodaj pa navadno dobimo vrednost, ki ustreza obvladujočemu deležu, zato na njo v primeru zaprtega podjetja apliciramo še odbitek za pomanjkanje tržljivosti ter za neobvladovanje, če iščemo vrednost manjšinskega deleža. Enako kot za metodo primerljivih kupoprodaj pa velja tudi za na sredstvih zasnovan način.\\
Osredotočimo se torej na dva najpomembnejša sklopa odbitkov. Za nas bo v prihodnje pomembnejši odbitek zaradi pomanjkanja tržljivoti, a ker velikost lastniškega deleža preko ravni obvladovanja v veliki meri določa stopnjo tržljivosti, moramo najprej analizirati vidik obvladovanja in z njim povezano prilagoditev. 



\subsection {ODBITEK ZA NEOBVLADLJIVOST}:\\

Vpliv možnosti obvladovanja podjetja na vrednost ocenjevanega deleža je vse prej kot zanemarljiv. Velikost odbitka je sicer v splošnem v primerih večjega deleža obvladljivosti (nad $50\%$) manjša kot pri manjšem deležu, podrobneje pa obstaja še več pomembnih razlogov za njegov nastanek in višino. Oglejmo si nekatere med njimi.
\begin{itemize}
\item Vpliv državnih predpisov\\
Ti se po svoji vsebini razlikujejo od države do države. Dejstvo je, da ponekod na pomembnejše odločitve v podjetju lahko vpliva šele delničar z več kot $50\%$ lastništvom, drugje pa je za to dovolj šele dvo-tretjinski deležnik, kar pomeni, da že delničar z več kot 1/3 delnic podjetja, lahko prepreči izvajanje določenih sprememb. V tem primeru je torej enotretjinski lastniški delež mnogokrat bolj cenjen od le malo manjšega, zato je odbitek vrednosti za neobvladljivost pri njem precej manjši. Velikost odbitka se lahko razlikuje tudi glede na zmožnost razrešitve vodilnih ljudi v podjetju s trenutnega položaja, s strani malih delničarjev. Tudi ta pravica delničarjev se namreč razlikuje od države do države, nenazadnje tudi od podjetja do podjetja.
\item Interni predpisi v podjetju\\
Podjetja si do določene mere lahko preoblikujejo predpise v sebi bolj ustrezne, čeprav se razlikujejo od državnih. V tem kontekstu je dober primer uvedba dvotretjinske večine za prevlado na glasovanjih, kljub državnim predpisom, ki določajo polovično podporo. 
\item Redčenje lastništva\\
Gre za spremembo porazdelitve lastništva v podjetju, ki je posledica na novo izdanih delnic ali pa prodaje delnic nazaj podjetju. Podjetje namreč s prodajo sebi lastnih delnic med delničarje (nove ali obstoječe), lahko zmede strukturo lastništva na način, ki lahko pomembneje vpliva na stopnjo obvladljivosti posameznega lastnika. 
\item Predkupna pravica\\
Nekatera podjetja v svojem statutu določajo predkupno pravico svojim obstoječim delničarjem. Ta jim daje možnost, da se prvi odločajo o nakupu lastnih ali na novo izdanih delnic podjetja, preden to lahko stori tretja oseba (ki trenutno ni lastnik). Prav ta ukrep naj bi delničarje branil pred pretirano spremembo lastništva. 
\item Način izvolitve vodilnih v podjetju\\
V nekaterih podjetjih, pravila pri glasovanju za določanje vodilnih dovoljujejo, da deležnik vse svoje glasove da isti osebi. S tem bi, v primeru posedovanja dovolj velikega deleža, lahko onemogočil prevlado želja manjšinskih delničarjev. Temu pravimo kumulativno voljenje.
\item Pogodbene omejitve\\
Določene omejitve pravic delničarjev, kot so neizplačevanje dividend delnic, izključenost pri delitvi premoženja ob likvidaciji podjetja, pomanjkanje pravic pri soodločanju o prevzemih podjetja ali prerazporeditvi vodstva,... lahko ključno vplivajo na višino odbitka, torej na vrednost ocenjevanega deleža.
\item Finančno stanje podjetja\\%%%
Mnoge pravice, povezane z obvladovanjem podjetja, so lahko označene kot ``ekonomsko prazne" preprosto zaradi finančnega stanja podjetja. V to štejemo pravico do soodločanja pri delitvah dividend, prodaji in nakupu sredstev ali delnic, nakupu drugih podjetij,... Podjetja na primer, ki dosegajo visoke donose pri poslovanju, ne uživajo tudi visokega pribitka za obvladovanje, če je večina pravic kontroliranja že porazdeljena med delničarje. 
\item Visoko regulirana panoga\\
Pri podjetjih, vključenih v panoge z visoko regulacijo s strani državnih predpisov, se odbitek za neobvladljivost lahko pojavi tudi pri $100\%$ lastniškem deležu. Razlog za to je, da tudi obvladujoči deležniki nimajo toliko pravic pri odločanju o podjetju, kot jih imajo lastniki manj reguliranih podjetij. Govorimo o sposobnostih državnih oblasti, da odloča o likvidaciji podjetja, njegovi prodaji, nakupu, začetku nove proizvodnje,... Če primerjamo torej manj regulirane panoge z bolj reguliranimi, je razlika v odbitku za neobvladljivost med manjšinskim in večinskim deležnikom pri prvih precej večja kot pri drugih. Razlog za to je namreč dejstvo, da tudi pri podjetjih, ki so bolj regulirana s strani države, niti večinski niti manjšinski lastnik nima popolnega nadzora nad vodenjem.
\item Sporazumi delničarjev o nakupu in prodaji delnic\\
V nekaterih podjetjih se lahko med delničarji oblikujejo dogovori o vrednotenju delnic pri njihovi prodaji ali nakupu bodisi neposredno med delničarji, bodisi med njimi in podjetjem. Primer takšnega dogovora je sklep, da se vse delnice (torej tudi manjšinski deleži) vrednotijo kot da imajo polno moč pri obvladovanju podjetja. V tem primeru se torej odbitek za neobvladljivost na njihovo vrednost ne aplicira.
\item Fiduciarna odgovornost\\
Včasih večinski lastniki ne morejo uživati vseh pravic, ki jim jih posedovani delež daje, ker imajo določene obvezosti do manjšinskih lastnikov. Pri tem mislimo na primer na enakovredno delitev dobička med vse delničarje. Tako obvladujoče kot neobvladujoče lastnike se tretira enako, zato je odbitek za neobvladljivost pri manjšinskem deležniku precej manjši kot sicer. Kljub temu pa je res, da je situacij, kjer bi se vse lastnike gledalo kot enakovredne, zelo malo.
\item Zasebno podjetje z javnimi vrednostnimi papirji\\
Deleži podjetij, ki imajo sicer vse delnice v internem lastništvu, hkrati pa so njihove obveznice javne, uživajo manjši odbitek neobvladljivosti. Razlog za to je dejstvo, da je podjetje z javnimi vrednostnimi papirji dolžno javno objavljati informacije, do katerih sicer manjšinski lastniki ne bi imeli dostopa. To so na primer informacije o finančnem poslovanju podjetja, transakcijah med deležniki in podjetjem... V takšnem primeru pa so tudi določene prednosti obvladujočega deleža izničene.
\item Podjetje z internim lastništvom, ki deluje kot javno lastninjeno\\
Podjetja, ki se kljub internemu trgu delnic obnašajo in delujejo v skladu s predpisi javno lastninjenih podjetij, so podvržena manjšemu odbitku neobvladljivosti. V tem primeru imajo namreč vsi delničarji enake informacije o podjetju in poslovanju, zato so razlike med večinskim in manjšinskim lastnikom manj očitne in manj ključne.
\item Obvladovanje podjetja je enakomerno razpršeno\\
V primeru, ko si na primer podjetje lasti 20 delničarjev, s približno enako velikimi deleži, se razlike med obvladujočim in neobvladujočim lastnikom zabrišejo. Vsi imajo približno enak vpliv na delovanje podjetja, zato je omenjeni odbitek manjši.
\item Vpliv presežnih sredstev\\
Količina sredstev, ki niso nujna za vsakdanjo delovanje podjetja, lahko vpliva na osnovo za apliciranje odbitka za neobvladljivost, v primeru, da jih podjetje v kratkem planira prodati in razdeliti dobiček v obliki dividend ali pa pretvoriti v dejavnost, ki bo prinašala dobiček v prihodnosti.
\end{itemize}
(Valuing a business, 2008)\\

V omenjenih postavkah je prilagoditev včasih obravnavana kot odbitek, drugič pa kot pribitek. Med slednjima pa obstaja povezava preko sledeče formule:
\begin{equation}
odbitek\ za\ neobvladljivost=1-\frac{1}{1+pribitek\ za\ obvladovanje}
\end{equation}

V splošnem pri vrednotenju manjšinskega deleža podjetja moramo v obzir vzeti dejstvo, da lastnik z manjšim deležem nima pravice razpolagati s presežnimi sredstvi podjetja, kakor je to v domeni večinskega lastnika. Posledično se pri opredelitvi prostega denarnega toka, znesek razlikuje za oba lastnika. Ob vrednotenju celotnega podjetja namreč stroške, ki jih oblikujejo večinski lastniki, štejemo v razpoložljiv znesek za njihove dividende. Na drugi strani pa ta vrednost za manjšinskega lastnika v resnici mora biti zmanjšana za vpliv večinskih lastnikov, ki razpolagajo z določenimi sredstvi podjetja. Prost denarni tok za neobvladujočega delničarja je torej avtomatsko manjši.


\subsection{ODBITEK ZARADI POMANJKANJA TRŽLJIVOSTI}:\\

Po podrobni opredelitvi odbitka za neobvladljivost, sledi analiza vpliva stopnje tržljivosti. Splošno znano dejstvo je, da je sredstvo, ki ga je moč hitreje prodati dalje, za investitorja vredno več, kot tisto z manj potencialnimi kupci. Deleži lastnikov v lastniško zaprtih podjetjih so predmet manjše tržljivosti v primerjavi z drugimi naložbami na trgu. Zato je za ocenjevalca vrednosti pomembno preučiti vpliv nižje tržljivosti sredstva in z njo povezane nelikvidnosti. V tem kontekstu pojem nelikvidnosti razumemo kot zmanjšanje sposobnosti lastnika celotnega podjetja, da pretvori svoja naložbena sredstva v denar hitro in s čim manjšimi stroški. Hkrati pa pomanjkanje tržljivosti opredeljuje isto nezmožnost za manjšinskega delničarja. Posledično bomo pri odbitkih zaradi manjše likvidnosti govorili o sposobnostih prodaje večjih deležev, sredstev podjetja, o združitvah,... Odbitki zaradi pomanjkanja tržljivosti pa se nanašajo na prodajo manjših deležev in posameznih delnic. \par
Razlogov za nastanek različnih stopenj obeh omenjenih odbitkov zaradi pomanjkanja tržljivosti je več:
\begin{itemize}
\item Prodajna pravica\\
Najmočnejši dejavnik, ki bi lahko v celoti izničil vpliv tega odbitka je pravica do prodaje deleža. Pri tem mislimo na deležnikovo prodajo ob določenem pogodbenem času ali pogojih, po dogovorjeni ceni. Ta pravica nam torej zagotavlja prodajo deleža v določenih okoliščinah, kar pomeni, da je strah nezmožnosti prodaje odvečen. 
\item Izplačila dividend\\
Delnice brez ali z nizkim zneskom izplačanih dividend so deležne večjega odbitka za pomanjkanje tržljivosti, kot tiste z visokimi donosi. Velja namreč, da dobiček lastnika delnice z večjimi dividendnimi izplačili ni odvisen le od kapitalskega donosa ob prodaji deleža. Več dobi že sproti v času imetja papirja, zato je takšna delnica precej bolj zaželjena med investitorji in lažje prodajljiva. 
\item Potencialni kupci\\
Obstoj sprejemljivega števila potencialnih kupcev deleža, ali celo enega z dovolj velikim potencialom nakupa, zmanjša velikost odbitka tržljivosti. 
\item Stopnja interesa\\
Za primer lahko vzamemo dejstvo, da se večji paket delnic težje proda. Razlog je manjše zanimanje potencialnih kupcev, saj so transakcije, povezane s tem nakupom, sorazmerno večje. Odbitek za netržljivost je torej večji, po drugi strani pa vemo, da je tovrstna prilagoditev zaradi večje obvladljivosti manjša. 
\item Verjetnost, da podjetje postane javno\\
V primeru, da je v bližnji prihodnosti zasebnega podjetja zelo verjetna (na primer) kotacija delnic na borzi, bo odbitek za pomanjkanje tržljivosti manjši. V kratkem se namreč pričakuje, da bo vrednost delnice narasla, ker bo le ta podvržena širšemu trgu in možnosti večjega povpraševanja. Res pa je, da je takšen razplet navadno negotov in težko napovedljiv.
\item Dostop in zanesljivost informacij\\
V primeru, da manjšinski deležnik nima dovolj informacij o poslovanju in stanju podjetja, je vrednost delnice z njegovega vidika manjša. Težje namreč oceni možnost prodaje svojega deleža dalje.
\item Pogoji prodaje restriktivnih delnic\\
V nekaterih podjetjih so oblikovani različni pogoji trgovanja z restriktivnimi delnicami, ki še dodatno omejujejo njihovo prodajo in nakup na področju cene, količine, potencialnih kupcev,... kar zaradi težjega trgovanja povzroči večji odbitek zaradi netržljivosti.
\item Karakteristike podjetja\\
Pretekla finančna stanja podjetja močno vplivajo na stopnjo tržljivosti, saj je v primerih konstantnega poslovanja tveganje nenadnega propada manjše in se delež lažje proda. Na drugi strani pa je podjetje, ki ima zelo razgibano finančno preteklost tako v smislu porasta dobička kot njegovega padca, bolj tvegano za naložbenika. S tem povezujemo tudi vpliv velikosti podjetja. Večji obseg poslovanja navadno pomeni stabilnejše podjetje, kar pa manjša odbitek. 
\end{itemize}
Kot že omenjeno, je obravnavani odbitek močno povezan s stopnjo obvladljivosti ocenjevanega deleža, vendar velja, da je uporaba statističnih podatkov kot osnove za določanje odbitka za pomanjkanje tržljivosti, bolj naklonjena ocenjevanju manjšinskega deleža. Pri vrednotenju zasebnih podjetij si torej ocenjevalci za osnovo, na katero bodo aplicirali odbitek, lahko izberejo vrednost restriktivnih delnic (delnice, s katerimi določen čas ni možno trgovanje na odprtem trgu) ali pa (ustrezneje) začetnih javnih ponudb (vrednost ponudbe za delnice pred postavitvijo njihove cene na borzi). In ko govorimo o manjšinskem deležu, se odbitek pri njegovi vrednosti giblje od $30\%$ do $50\%$ tržne cene delnice.

\subsubsection{Določanje odbitka na podlagi vrednosti restriktivnih delnic}:\\

Takšen način vrednotenja za osnovo privzema vrednost restriktivne, na primer pisemske delnice, ki se od kotirane delnice na borzi razlikuje le po tem, da je z njo določeno obdobje nemogoče javno trgovati. Tržljivost je torej edina razlika med pisemsko in javno tržljivo delnico istega podjetja, zato je njen vpliv toliko lažje razbrati.\\
Javna podjetja običajno izdajo pisemske delnice v primerih prevzemanja drugih podjetij ali dviga lastnega osnovnega kapitala. Njihova registracija preko javne družbe je namreč nepraktična z vidika porabljenega časa in stroškov izdaje. Poleg tega, je v tem primeru podjetje dolžno objavljati določene podatke o trgovanju z omenjenimi delnicami, tudi če le to poteka izključno znotraj internega trga delnic. \par
Uporaba primerjave z vrednostmi restriktivnih delnic je s strani ocenjevalcev vrednosti zelo pogosta. Kljub temu pa velja, da je zaradi dejstva, da bo lastnik restriktivne delnice lahko po določenem obdobju delnico prodal na organiziranem trgu, le ta danes za njo pripravljen plačati več kakor za delež v lastniško zaprti družbi. Zasebni delež ima namreč zelo malo možnosti, da bo kdaj podvržen trgovanju na borzi. Prav zato pa nam odbitek, dobljen preko primerjave z restriktivnimi delnicami še vedno ne daje realne vrednosti. Zaradi te omejenosti raziskav restriktivnih delnic, so ocenjevalci pričeli primerjati cene trgovanj z delnicami, ki so bile realizirane pred prvo javno ponudbo s cenami delnic istega podjetja, ko so bile le te že uvrščene na borzo.

\subsubsection{Določanje odbitka na podlagi vrednosti pred prvo javno ponudbo}:\\

Kot pomoč pri oceni višine odbitka se lahko privzame tudi vrednost delnice pred prvo javno ponudbo. To je vrednost, ki še ni obremenjena z delovanjem trga in podvržena tržljivosti, kakor delnica, ki že dalj časa kotira na borzi. Odbitek je torej po tej metodi vrednoten toliko, kot je razlika med ocenjeno vrednostjo delnice pred njenim pojavom na borzi ter njeno vrednostjo pri kotaciji na javnem organiziranem trgu.  \\


Prav tako kot pri vrednotenju manjšinskega deleža podjetja, pa moramo odbitek za pomanjkanje tržljivosti aplicirati tudi na obvladujoči delež. Pri tem, kot omenjeno, uporabljamo izraz ``nelikvidnost", ki označuje nezmožnost lastnika podjetja hitro in s čim manjšimi stroški unovčiti svojo lastnino. V primerjavi z lastnikom posamezne delnice oziroma manjšega deleža podjetja, obvladujoči lastnik v lastniško zaprtem podjetju precej težje proda svoj delež. Ker pa je o določanju odbitka zaradi pomanjkanja tržljivosti obvladujočih deležev zelo malo empiričnih podatkov, je pri preučevanju potrebno upoštevati posebnosti vsakega posameznega primera.%%% Na znanje moramo pri tem vzeti dejstvo, da pomanjkanje tržljivosti pri obvladujočem deležu povzroča znatno manjši odbitek kot pri manjšinskem.%%
Prodaja večinskega deleža podjetja je namreč zahteven, trajen in negotov postopek, kar še toliko bolj velja za deleže zasebnih podjetij. Postopek unovčitve obvladujočega deleža podjetja zajema pridobitev ponudb za nakup deleža in zasebno prodajo celotnega podjetja ali vsaj večinskega deleža. Lastnik takšnega deleža mora torej pri prodaji obravnavati naslednja dejstva:
\begin{itemize}
\item Negotovost trajanja časovnega obdobja, v katerem bo prodaja izvršena\\
Do določene mere sicer lahko daljše obdobje čakanja kompenzirajo donosi deleža, ki jih lastnik v tem času še vedno prejema.
\item Stroški, ki nastanejo zaradi postopka prodaje\\
S tem mislimo na stroške postopka pridobitve potencialnih kupcev in njihovih zahtev, stroške za pridobitev potrebne dokumentacije in pogajanj glede jamstev ter administrativne stroške za delo z odvetniki, računovodji, potencialnimi kupci ali njihovimi predstavniki,...
\item Tveganje povezano z dejansko doseženo vrednostjo posla\\
Stopnja takšnega tveganja je navadno visoka, saj je posel ovrednoten le s pričakovano vrednostjo, poleg tega pa na dolgotrajen postopek trgovanja lahko vplivajo tako zunanji kot notranji dejavniki. Dalje lahko dodamo, da vedno obstaja tveganje, da se posla sploh ne da skleniti, ne glede na pričakovano ceno. Velja namreč, da kljub običajni dovzetnosti javnih trgov za transakcije z delnicami različnih podjetij iz različnih industrij, nekatera zasebna podjetja nikoli ne bodo javno sprejeta.
\item Oblika postopka transakcije\\
%%%%%%%%%%%
Čeprav je postopek prodaje že zaključen, lahko kupec plačilo deloma izvrši mnogo kasneje ali pa v obliki delnic, ki so navadno restriktivne. V tem primeru vrednost delnice zasebnih podjetij (kot razloženo prej) ni enakovredna restriktivni delnici odprto lastniških družb.
%%%%%%%%%%%
\item Nezmožnost predhodnega izplačila\\
Celo prodaja obvladujočega deleža zasebnih podjetij bankam običajno ne predstavlja dovolj gotovega posla. Prav zato lastniku, ki čaka na prodajo, ne bodo vnaprej izplačale predvidenega zneska, ki naj bi ga dobil pri prodaji svojega deleža, tudi če ta denar nujno rabi.
\end{itemize}
Navedli smo že dva načina, kako najlažje dobimo primerljivo vrednost za določanje višine odbitka pri manjšinskem deležu, sedaj pa se osredotočimo še na obvladujočega. Odbitek zaradi pomanjkanja likvidnosti tako lahko določimo glede na:
\begin{itemize}
\item Ceno, ki jo prodajalec deleža doseže na javnem trgu pri prvi ali drugi ponudbi 
\item Ceno, doseženo pri zasebni prodaji lastniško popolnoma zaprtih podjetij
\item Nakupe obvladujočih deležnikov javno financiranih podjetij
\end{itemize}

Da imata lastništvo podjetja in njegova odprtost trgu znaten vpliv na njegovo vrednost, je vidno tudi iz razlik vrednosti v prejšnjem poglavju omenjenih multiplikatorjev med javnimi in zasebnimi podjetji. Navadno so namreč povprečni multiplikatorji za nakup zasebnega podjetja precej nižji od tistih, za javno podjetje. Razlogov za tovrstno razliko pa je več:
\begin{itemize}
\item Izpostavljenost trgu
\item Kvaliteta in zanesljivost računovodskih podatkov 
\item Učinek velikosti
\end{itemize}
Podatki o javnih podjetjih in njihovih cenah delnic so vsakodnevno objavljeni v raznih medijih in predstavljajo lahko dostopne informacije za potencialne vlagatelje. Tovrstna podjetja so do določene mere tudi dolžna objavljati nekatere podatke na spletnih straneh in na raznih srečanjih z mediji. Nasprotno pa to ne velja za zasebna podjetja, ki svojih podatkov ne razkrivajo javnosti, zaradi česar se potencialni kupci težje odločajo o njihovem nakupu. V izboru zasebnih podjetij zaradi pomanjkljivih informacij namreč ne znajo izbrati najbolj ustreznega. Manjša izpostavljenost trgu torej preko odbitka zaradi pomanjkanja tržljivosti povzroča nižjo vrednost podjetja.\\
Podjetja, ki vodenje računovodskih storitev in upravljanje z delnicami prepustijo posebnim družbam, ki so za to specializirane, se v javnosti pojavljajo z zelo zanesljivimi in urejenimi podatki. Posledično pa so za vlagatelje bolj zanimiva, saj se njihov nakup zdi manj tvegan. Večje zanimanje za nakup tako prinaša večjo tržljivost in višjo vrednost deleža.\\
Velikost podjetja pa je po mnenju nekaterih analitikov še tretji dejavnik, ki pripomore k razlikovanju multiplikatorjev vrednotenja podjetij. Zasebna podjetja so namreč navadno manjša od javnih in imajo hkrati nižje vrednosti. \\











%%%%%%%%%%%%%%%%%%%%%%%%%%%%%%%%


\section{\textbf{KRATKA PREDSTAVITEV PODJETJA X}}:\\

Podjetje X je eno večjih slovenskih industrijskih podjetij, ki deluje že od leta 1946, njegova glavna dejavnost pa je proizvodnja električnih motorjev in komponent iz laminatov, aluminija, termo plastike,... Njihovi izdelki se uporabljajo večinoma za vgradnjo v vakuumske enote, pa tudi na področju bele tehnike, prezračevanja, avtomobilske proizvodnje, medicine,...
Največji izvozni trg jim predstavlja Nemčija, v manjši meri pa sodelujejo tudi z Madžarsko, Švedsko, Poljsko, Italijo, Avstrijo, Romunijo,... Zgovoren podatek o njihovi uspešnosti je tudi dejstvo, da je, kot že omenjeno zgoraj, v kar šest od desetih prodanih sesalnikov v Evropi, vgrajen motor, ki je proizveden v podjetju X. V svetovnem merilu sodelujejo s podjetji kot so Elektrolux, Philips, Rowenta, Stihl, Husqvarna, Samsung,... poleg štirih proizvodnjih enot v Sloveniji pa ena obratuje tudi na Kitajskem.\par
Podjetje X ima zelo pestro zgodovino, tako z vidika menjave področij delovanja, kakor tudi načina in organizacije vodstva. Prvih 12 let od ustanovitve se je ukvarjalo predvsem s predelavo in izdelavo kovinskih izdelkov. Po še dveh preimenovanjih se šele leta 1994 oblikuje v delniško družbo, v letu 2010 pa večjo prelomnico za podjetje pomeni preoblikovanje iz delniške družbe v družbo z omejeno odgovornostjo. Posebnost podjetja je, da ostaja v lasti zaposlenih, bivših zaposlenih in upokojencev, kar pa je v slovenskem gospodarskem prostoru edinstven primer. Leta 1998, v času množičnega lastninjenja slovenskih podjetij, so delavci podjetja X reorganizirali združenje notranjih delničarjev v družbo pooblaščenko. To so storili iz strahu pred domnevnim sovražnim prevzemom podjetja s strani tuje družbe, ki naj bi podjetje želela prevzeti zaradi že takrat velikega tržnega deleža v Evropi (20\%). Prihodnost pooblaščenke je bila sicer zelo negotova in mnogi so obetali propad podjetja, v primeru da ta ne dobi močnega strateškega partnerja. Ampak strah delavcev je bil močnejši, zato je skupina glavnih zastopnikov delničarjev pričela odkupovati delnice vseh ostalih in tako večati svoj lastniški delež. Posledično so na odločilni skupščini ohranili vlogo najmočnejšega lastnika matičnega podjetja in s tem zavarovali lastne interese ter preprečili sovražni prevzem. Delnice je pooblaščenka v veliki meri predala matičnemu podjetju, ki jih je nato razdelilo svojim zaposlenim. Tako danes lastniki podjetja v 92,36\% deležu ostajajo zaposleni, bivši zaposleni in upokojenci. Ostalih 7,64\% pa ostaja delnic vplačanih z denarjem, torej ne nujno del pooblaščenke. Tudi te z dohodkom iz dejavnosti, ki jih opravlja, pooblaščenka odkupuje in tako še veča delež lastništva matičnega podjetja. \\
Proizvodnja motorjev, ki so glavni program podjetja X, se je povečala za 50\%, močno pa se je povečala tudi prodaja v ZDA. Dobiček se je povečal za 4 krat, investicije v novo tehnologijo pa za 4,5 krat. V tem času so ohranili prav vse kupce in pridobili nove na tržiščih, na katerih podjetje do tedaj še ni bilo prisotno.



\subsection{OSNOVNO O DELNICAH PODJETJA X}:\\

Kot razloženo je torej podjetje X delniška družba, katere večinski lastniki so zaposleni ($52,66\%$), upokojenci ($30,51\%$) in bivši zaposleni ($9,06\%$). Podjetje ima tudi $0,13\%$ lastnih delnic, ki jih predvsem v zadnjih obdobjih skuša razporediti izključno v roke zaposlenih.\\
V celoti so delnice družbe razdeljene na tiste z oznako A in tiste z oznako B. Imetnik delnice A ima pravico do enega glasu na skupščini, do sorazmernega izplačila dividend ter do sorazmernega dela iz ostanka stečajne ali likvidacijske mase, v primeru stečaja podjetja. Glasovalne pravice vsakega delničarja so omejene, in sicer na 2 odstotka izdanih glasovalnih delnic, ne glede na dejanski delež njegovega lastništva.\par
Posebnost delnic podjetja X je, da so vinkulirane, kar pomeni, da je za njihov prenos lastništva zunaj kroga obstoječih delničarjev potrebno posebno soglasje nadzornega sveta. Na ta način se ohranja zaprt krog lastnikov podjetja. V primeru, da so delnice z oznako A ponujene obstoječim delničarjem, imetnikom delnic z oznako A ter v 30 dneh ni prišlo do prodaje, predkupno pravico dobi družba (za sklad lastnih delnic). Če ta v istem obdobju (30 dni) te pravice ne uveljavi, lahko delničar šele s soglasjem nadzornega sveta delnice ponudi tretji osebi.
Za delnice z oznako B, ki so pridobljene na podlagi stvarnega vložka saj jih je delničar vplačal z gotovino pa velja, da se lahko prosto prenašajo tudi izven kroga obstoječih delničarjev. Poleg tega njihova prodajna cena ni vezana na upoštevanje določb statuta družbe, kot to velja za delnice A. V primeru, da družba ne koristi predkupne pravice delnic z oznako B, lahko delničar le te brez soglasja prosto proda tretji osebi.\par
Pri prodaji velja, da ceno delnic, ki se uporablja za trgovanje delnic z oznako A, določi nadzorni svet, pri čemer upošteva knjigovodsko, ocenjeno in tržno vrednost delnice. Konec leta 2017 je osnovni kapital družbe sestavljalo 514.215 delnic z nominalno vrednostjo 4,1729 EUR. Podjetje jih je v preteklem letu kupovalo po ceni od 5,50 do 7,00 eurov na delnico, na skupščini leta 2018 pa so ceno omejili med 10 in 20 euri na delnico.


\subsection{ANALIZA MAKROEKONOMSKEGA OKOLJA}:\\

Kot omenjeno je torej podjetje X elektroindustrijsko podjetje, katerega glavna dejavnost je proizvodnja elektromotorjev, generatorjev in transformatorjev. V sklopu vseh dejavnosti ta spada med predelovalne, ki so najobsežnejše. Podjetje se nahaja na območju Gorenjske regije, ki spada na četrto mesto po rezultatih poslovanja (merilo so število družb, zaposlenih, skupni prihodki) vseh gospodarskih družb v Sloveniji. Glede na to, da velik del njegovih dobaviteljev predstavljajo slovenske družbe, kupci pa so v veliki meri tuje, je za našo napoved prihodnjih denarnih tokov zelo pomembno stanje gospodarstva tako pri nas kot tudi v tujini.

\subsection{GOSPODARSKA GIBANJA V SLOVENIJI}:\\

Predelovalne dejavnosti v Sloveniji po podatkih za leto 2017 pokrivajo največji delež ($53,9\%$) čistih prihodkov od prodaje na tujih trgih, hkrati pa prevladujejo tudi po številu zaposlenih. 
Na slovenskem prostoru je tako imela po podatkih Gospodarske zbornice Slovenije za leto 2017, elektronska in elektroindustrija (EEI) $20,5\%$ delež izvoza in je ustvarila $17,92\%$ dodane vrednosti slovenske predelovalne industrije. Čisti prihodki od prodaje za EEI že od leta 2009 naraščajo, prav takšen pa je tudi trend gibanja deleža prodaje na tuje trge. Dodana vrednost na zaposlenega po padcu v 2009 strmo narašča, stroški v njej pa se praviloma znižujejo, z eno izmed izjem tudi v letu 2017 (povišanje za $1,4\%$). Neto čisti dobiček skozi leta počasi narašča, prav tako pa praviloma tudi kazalnik ROE (delež dobička v kapitalu).\\
Proizvodnja električnih naprav je v letu 2017 znotraj vseh predelovalnih dejavnosti zabeležila največji skupni prihodek, in sicer $12,2\%$, z višino neto čistega dobička pa se uvršča na drugo mesto, takoj za proizvodnjo farmacevtskih surovin in drugih preparatov. Tako skupni prihodek kot neto čisti dobiček sta višja kot leto poprej, čista izguba pa upada.\\

Po podatkih Zbornice elektroindustrije Slovenije bo v prihodnjih letih povpraševanje še vedno poganjal izvoz, ki bo sicer prednjačil pred domačo porabo in javnimi investicijami, a se bo njegova rast počasi umirjala. To naj bi bila predvsem posledica upočasnitve rasti tujega povpraševanja in odsotnosti nekaterih dejavnikov v avtomobilski industriji, ki so izvoz povečevali v zadnjih letih. Hkrati se bodo stroški dela postopno zviševali, zato izboljševanja izvozne konkurence ni pričakovati. V obdobju 2018-2020 naj bi po napovedih UMAR rast investicij ostala visoka. Podjetja naj bi investirala zaradi rasti povpraševanja, zasebne investicije pa bodo posledica optimizma med potrošniki in ugodnega gibanja na trgu dela, čeprav bodo kasneje v prihodnosti zaradi umirjanja rasti zaposlenosti precej nižje. Višanje cen nafte in storitev naj bi privedlo do manjšega povečanja inflacije na dobra $2\%$.

\subsection{GOSPODARSKA GIBANJA V EVROPI IN PO SVETU}:\\

Splošni trendi tako na evropskih trgih kot tudi drugod po svetu, so usmerjeni v blažjo rast elektroindustrije kot v zadnjih letih. Pri tem je največji napredek zaznan v avtomobilski industriji, s povečano prodajo gospodarskih in osebnih vozil. Negativni trendi pa se znotraj elektroindustrije pojavljajo predvsem zaradi nadaljnjega upočasnjevanja trga BRICS (Brazilija, Rusija, Indija, Kitajska, Južna Afrika), ki v veliki meri izhaja iz odnosov med Rusijo in Ukrajino in devalvacije Ruskega rublja. %Prav tako pa se je znižala vrednost Angleškega funta, kar draži izvoz v Anglijo.%
Kljub temu se v prihodnosti pričakuje zmerna rast v večini industrijskih držav, verjetnost pa obstaja, da bodo dodatno zavoro na pozitivna gibanja prinesli dogodki v zvezi z brexitom in novo mednarodno trgovinsko politiko predsednika Donalda Trumpa. Napoved za prihodnost s strani evropske organizacije IDEA (International Distributors of Electronic Association) torej vključuje pozitivno, a počasnejšo rast panoge, kar so podkrepili s podatki v grafu:
%slika%

%do tu je ok

%od tu naprej je spet ok

\subsection{FINANČNA ANALIZA POSLOVANJA PODJETJA X}:\\

Predstavila bom glavne lastnosti gibanj osnovnih postavk bilance stanja podjetja X v preučevanih 5 letih, torej v obdobju 2013 do 2017. Vse podatke sem pridobila iz javno objavljenih letnih poročil podjetja, za lažjo analizo pa sem zraven vzela še tri podjetja, s katerimi sem primerjala naše izbrano podjetje. To so Kolektor Sikom, Bosch Rexroth d.o.o. in EBM-Papst Slovenija, ki izhajajo iz iste panoge kot podjetje X. Po svoji usmerjenosti v izvoz v tujino so si zelo podobni, a po vrednosti sredstev precej različni.\\

\subsubsection{Prihodki od prodaje}:\\

Če si najprej ogledamo prihodke od prodaje našega podjetja vidimo, da so v preučevanih petih letih spodbudno naraščali, in sicer s povprečno stopnjo 114\%. To je predvsem posledica vse večje proizvodnje in prodaje, podjetje je razširilo svojo ponudbo in se v veliki meri pričelo ukvarjati tudi z avtomobilsko industrijo, s čimer so si povečali trg. Ker je podjetje izrazito izvozno usmerjeno, se krepitev gospodarstva v tujini in okrevanje podjetij po krizi vidno močno odraža na izboljšanju njegovih rezultatov poslovanja. Naraščali so tudi drugi prihodki, kot so prihodki iz poslovnih terjatev in drugi usredstveni lastni proizvodi in storitve. Ob primerjavi rasti prihodkov našega podjetja in preostalih treh lahko opazimo, da prihodki od prodaje pri nobenem ne naraščajo s tako visoko stopnjo. 
%graf
Na grafu kjer primerjamo rasti prihodkov vseh štirih podjetij, je edino podjetje Bosch Rexroth tisto, ki v zadnjem letu 2017 preseže rast prihodkov podjetja X, po upadu v letu 2015. Res pa je, da so vsa ostala tri podjetja naklonjena trendu zviševanja stopnje rasti, kar je za v prihodnje zelo spodbudno, zato se lahko kmalu srečajo na nivoju podjetja X.\\

\subsubsection{Stroški}:\\

Celotne stroške podjetja X sem razdelila na glavne postavke, ki so stroški blaga, materiala in storitev, stroški dela, prevrednotovalni poslovni odhodki in amortizacija.\\
Stroški blaga, materiala in storitev so v preučevanih petih letih naraščali prav tako kot prihodki, s povprečno stopnjo 114\%. Glavni delež predstavljajo stroški materiala, ki so povečani predvsem zaradi povečane proizvodnje in nabave materiala, dobave boljših in zahtevnejših materialov. Znotraj prihodkov od prodaje se je delež stroškov materiala do leta 2015 zmanjševal (iz 54\% na 50\%), nato pa se je trend začel slabšati in do leta 2017 zopet dosežemo 53\%. Zmanjševanje je predvsem posledica optimiziranja zalog in nabave surovin, nato pa je povečanju verjetno sledilo tudi zaradi novo načrtovane proizvodnje izdelkov, ki se še niso prodali, kar se vidi tudi na precejšnjem povečanju zalog v zadnjih dveh letih.\\
Naslednja pomembna postavka so stroški dela, ki predstavljajo dobrih 20\% prihodkov od prodaje, njihov delež pa se v trendu znižuje. Čisti dobiček na zaposlenega je v petih letih zrastel kar na 301\% vrednosti, in sicer na 8573 evrov, kljub temu da podjetje zaposluje v precej veliki meri, hkrati pa se stroški na zaposlenega postopoma večajo po nekaj odstotnih točk na leto. To je sicer tudi posledica vse večjih izplačevanj nagrad in stimulacij.\\
Amortizacija pri podjetju X je precej stalna glede na gibanje prihodkov, in sicer predstavlja njihov 5\% delež, poleg tega tudi prihodnosti glede na investicije ni pričakovati prevelikih sprememb. To bom omenila predvsem kasneje, pri analizi napovedi za prihodnjih 5 let.
Ob primerjavi z ostalimi tremi podjetji opazimo, da je vsem skupna večja rast stroškov kot prihodkov za leto 2016. Sklepamo lahko, da gre za večje dobave materiala, investiranje in povečanje proizvodnje, ki sovpadajo z rastjo in razvojem tujih trgov. Iz podatkov vidimo, da naše podjetje vse bolje obvladuje stroške, saj se njihov delež v prihodkih manjša, in sicer se v skupnem zmanjša za 5\%, na 92\% prihodkov od prodaje. Pri ostalih treh podjetjih pa je sprva videti, da daleč najbolje stroške obvladuje Kolektor Sikom, saj ti predstavljajo pod 90\% prihodkov, Bosch Rexroth in EBM-Papst Slovenija pa dosegata vrednosti blizu 100\%. Vendar tu moramo upoštevati tudi same prihodke in njihovo rast, drži namreč, da je prav rast prihodkov podjetja Kolektor Sikom, najnižja med vsemi.\\

\subsubsection{Dobiček in EBITDA marža}:\\

Dobiček iz poslovanja podjetja X, po davkih, je v 5 letih narasel za 8 milionov evrov, njegova rast pa se upočasnjuje. To gre pripisati predvsem povečanju celotnih stroškov, saj se stopnja njihove rasti v letih 2016 in 2017 čisto približa rasti skupnih prihodkov. Rast čistega dobička tako iz 218\% v letu 2014 pade na 106\% v letu 2017.
EBITDA ali dobiček iz poslovanja pred amortizacijo je eden od pokazateljev poslovne uspešnosti podjetja. Ker je amortizacija leračunovodski poseg (knjižba), ki nima nobenega vpliva na denarni tok podjetja, se pri izračunavanju EBITDA prišteva nazaj k dobičku iz poslovanja. Pri podjetju X se v povprečju njena rast giblje okoli 122\%, trend upadanja pa je predvsem posledica upada porasta poslovnega izida iz poslovanja. Rast EBITDA se sicer znižuje tudi na področju celotne elektroindustrije, a je vseeno nižja od rasti pri podjetju X. \\
%%% primerjava ebitda gibanja X in ostalih treh podjetij??


\subsubsection{Sredstva in neto obratni kapital}:\\

Na strani sredstev večinski delež do leta 2016 predstavljajo kratkoročna sredstva, v omenjenem letu pa v podjetju izvedejo večje investicije v osnovna opredmetena sredstva, natančneje v proizvajalne stroje in opremo. V ta namen se tudi zadolžijo pri večjih slovenskih bankah, kar se pozna tudi na povečanih dolgoročnih obveznostih na strani pasive v bilanci stanja. Sicer imajo v sklopu kratkoročnih sredstev glavno vlogo poslovne terjatve, ki se nanašajo predvsem na kupce na tujih trgih, saj je podjetje izrazito izvozno usmerjeno. \\
V deležu prihodkov, so opredmetena osnovna sredstva v analiziranih letih predstavljala dobrih 20\% prihodkov, porast preko 30\% v zadnjih dveh letih pa je predvsem posledica zgoraj omenjenih povečanih investicij.\\
Zaloge so do leta 2017 predstavljale od 13 do 15\%, pri čemer je padec viden v letu 2015, ko se je povečala prodaja, proizvodnja pa temu povečanju še ni povsem sledila. Delež zalog se je tako zmanjšal, že v naslednjem letu pa zopet povečal, saj so dobavili nov material in pospešeno pričeli s proizvodnjo. Enak trend sem omenila že zgoraj pri porastu tako celotnih stroškov kot tudi prihodkov od prodaje. Politika podjetja je sicer na dolgi rok usmerjena v vitko proizvodnjo, kar pomeni nagnjenost k zmanjševanju kopičenja zalog.  \\
Na strani terjatev so predvsem pomembne kratkoročne terjatve do kupcev, ki predstavljajo v povprečju okoli 18\% prihodkov od prodaje. Ob povečani prodaji v letu 2015 je viden nekoliko višji delež (20\%), ki pa kmalu pade nazaj na 17\%. To kaže na precej učinkovito konvertiranje terjatev v denarna sredstva, kar nam prikazuje tudi doba obračanja terjatev, ki v petih letih iz 90 dni upade na 54. \\

Če dobičkonosnost sredstev preko kazalnika ROA primerjamo znotraj vseh 4 podjetij vidimo, da so najbolj dobičkonosna sredstva podjetja Kolektor Sikom, njihov ROA koeficient pa v zadnjem letu po 5 letih naraščanja ujame tudi podjetje X.\\

Zaloge in terjatve do kupcev sta najpomembnejši postavki pri izračunavi neto obratnega kapitala podjetja. Ta nam prikazuje presežek kratkoročnih sredstev nad kratkoročnimi obveznostmi. Večja kot je njegova vrednost, kasneje po nakupu so produkti podjetja res plačani s strani kupcev in več zalog se v podjetju kopiči. Zato je z vidika denarnih tokov dobro dosegati čim nižje vrednosti neto obratnega kapitala. Pri podjetju X lahko vidimo, da je njegova vrednost do leta 2015 rasla, nato pa do leta 2017 pričela upadati. Upad je predvsem posledica boljšega upravljanja z zalogami, porast v 2015 pa je povzročila povečana prodaja in s tem več kratkoročnih terjatev do kupcev, ki so jih v prihodnjih letih uspešno omejili.\\



\subsubsection{Financiranje}:\\

Poznamo dve vrsti financiranja podjetij, in sicer financiranje z dolžniškim kapitalom, ki je s strani podjetja bolj ugodno in pa z lastniškim kapitalom, ki predstavlja manjše tveganje. Najbolj racionalno je torej, da podjetje ohranja sorazmeren delež obeh vrst financiranja. Pri podjetju X vidimo, da je se financira predvsem s strani lastniškega kapitala, in sicer njegov delež predstavlja dobrih 70\% financiranja. Najbolj je delež narasel v letu 2015, ko so finančne obveznosti (tako dolgoročne kot kratkoročne) upadle glede na preteklo leto, saj so bila bančna posojila preteklo leto precej višja, kar nekaj pa so jih v letu 2015 že tudi poplačali.
%%% primerjava gibanja dolga X in ostalih treh podjetij??

Oglejmo si še kazalnik neto dolga znotraj EBITDA vrednosti, ki je postal sinonim za tveganost podjetja. "V smiselno razmerje postavi zadolženost in dobičkonosnost družbe z namenom ocenjevanja zmožnosti družbe poravnavati svoje finančne dolgove v prihodnosti, in sicer ob predpostavki, da družba ohrani isti obseg poslovanja in dobička. Višji Neto dolg na EBITDA pomeni manjšo sposobnost družbe odplačevati dolgove ter obenem višjo mero tveganja za vlagatelja oziroma kreditorja.
Velja standard, če vrednost kazalnika preseže 4 ali 5, obstaja velika verjetnost da bo družba v prihodnje s težavo poravnavala svoje finančne obveznosti ter hkrati ne bo zmožna pridobivati novih finančnih virov, ki so nujni za rast in razvoj podjetja. To je splošno načelo, čeprav je višina kazalnika povsem različna med panogami. "\\ 
(http://www.financnislovar.com/definicije/neto-dolg-ebitda.html)\\
Pri našem podjetju X se kazalnik giblje malo nad 1, kar nakazuje na nizko tveganost podjetja in njegovo privlačnost za vlagatelja.\\
%%% mogoče tudi neto dolg/EBITDA znotraj podjetij?
%%% preveri izračun neto dolga


%%%%%% do tu je ok

%%%%%%%%%%%%%%%%%%%%%%%%%%%%%%%%%%%%%
% od tu je spet ok

\subsection{NAPOVED POSLOVANJA PODJETJA X V IZBRANEM OBDOBJU IN IZRAČUN FCF}:\\

Kot obdobje, na katerega bom projecirala ugotovitve iz preučevanega petletja (2013 - 2017), sem vzela naslednjih 5 let, ki sledijo, torej obdobje 2018 - 2022. Za ta leta sem najprej na podlagi analize poslovanja opisane zgoraj, napovedala poslovni izid, nato pa ga diskontirala na datum vrednotenja, torej na 31.12.2017. Pri tem sem uporabila diskontno stopnjo, katere izračun bom kasneje podrobneje utemeljila, saj je bilo ravno določanje odbitkov in pribitkov diskontnega faktorja eden ključnih delov moje raziskave.\\

\subsubsection{Napoved prihodkov od prodaje}:\\

Za pet prihajajočih let sem stopnjo rasti prihodkov od prodaje nastavila na 1,12, kar je za 3 odstotne točke nižje od rasti v letu 2017. Takšno rast sem napovedala, ker je podjetje izrazito izvozno usmerjeno (88\% izvoza v tujino), po napovedih o rasti gospodarstva pa naj bi se rast na enem največjih trgov podjetja X (Nemčija), v prihodnosti nekoliko umirila. Enako velja tudi za ostale tri glavne trge, to so Avstrija, Romunija in Madžarska. Na drugi strani pa so napovedi za avtomobilsko industrijo, ki predstavlja vedno večji delež proizvodnje podjetja X, pozitivne. V grafu sem prikazala, kako je prodaja na področju avtomobilske industrije v podjetju naraščala v zadnjem petletju. Tako se je iz slabih 7\% v letu 2013 do leta 2017 povzdignila na slabih 16\%, kar je ob hkratnem velikem povečevanju skupnih prihodkov od prodaje še posebej spodbudno. V primerjavi sem zajela glavna tri področja proizvodov (avtomobilska industrija, ventilatorji in sesalne enote) in razvidna je rast vseh treh. Ker je Nemčija, kot omenjeno, glavni izvozni trg podjetja X, pričakujejo pa več kot 2\% letno rast trga avtomobilske industrije, predpostavljam še naprej pozitivno rast prihodkov prodaje podjetja X. Rast drugih prihodkov sem v prihodnje napovedala usklajeno z letom 2017, saj je stopnja rasti celotnih poslovnih prihodkov predvsem odvisna od rasti prihodkov od prodaje, večjih finančnih prihodkov iz poslovnih terjatev pa podjetje ne pričakuje.\\


\subsubsection{Napoved odhodkov}:\\

Odhodke podjetja sem napovedovala na podlagi njihovega deleža v poslovnih prihodkih v preučevanih 5 letih. Kot glavna postavka so tu stroški materiala, storitev in blaga. Njihov delež sem iz 62\% v 2017 do leta 2022 povečala na 66\%, predvsem na podlagi napovedi povečanja stroškov materiala in surovin. Stroški storitev in nabavna vrednost prodanega blaga po predvidevanjih ostajajo enaki preteklim letom.\\
Na drugi strani pomemben del odhodkov predstavljajo stroški dela, katerih delež po mojih napovedih v prihodnosti upade. Napovedala sem 5\% rast stroškov na posameznega zaposlenega, kar je povprečna rast preteklega petletja. V prihodnje pričakujemo nekoliko večjo produktivnost, zato sem temu prilagodila tudi rast prihodkov na zaposlenega. Stopnjo sem nastavila na malo več od povprečja zadnjih petih let, in sicer 9\%. Od tod pa sem pridobila število zaposlenih, ki je potrebno za doseg zastavljenih ciljev in napovedanega poslovnega izida. Podjetje je usmerjeno k precejšnjemu zaposlovanju in krepitvi kadra tako v proizvodnji kot tudi v razvoju. Z izračunom pridobljena stopnja zaposlovanja se mi zdi smiselna, saj ima podjetje v sklopu širjenja proizvodnje v Škofji Loki in tudi na drugih mestih v Železnikih še dovolj kapacitet za nove delavce. Tako produktivnost ob tolikšnem številu delavcev ne bi bila ogrožena. Dalje sem stroške na zaposlenega množila z izračunanim potrebnim število zaposlenih in dobila celotne stroške dela.  \\
Ob seštevku obeh napovedanih postavk stroškov lahko vidimo, da napoved v prihodnjih 5 letih predvideva 92 odstotni delež stroškov v prihodkih, kar je praktično enako zadnjim trem preučevanim letom (2015, 2016 in 2017). Glede na pretekla leta torej podjetju pripisujem usklajeno razmerje med stroški in prihodki tudi v prihodnje.\\


\subsubsection{Napoved investicij}:\\

Podjetje po povečanju investiranja v zadnjih dveh letih proučevanega petletja v prihodnosti napoveduje zmerne investicije. Zaključujejo namreč z obnavljanjem in širitvijo proizvodnih prostorov na Trati v Škofji Loki, del pa je bil namenjen tudi avtomatizaciji proizvodnje, povečanju produktivnosti in izboljšanju delovnih pogojev. Delež investicij sem zato v prihodnje letno nastavila na 7 do 8\%, kar je malo pod povprečjem preteklih let. Največji delež bodo še vedno predstavljale investicije v proizvajalne naprave in stroje, in sicer slabo polovico, gradbeni objekti in druga oprema pa ostaneta pri četrtini investicij v opredmetena osnovna sredstva.\\
Investicije v neopredmetena dolgoročna sredstva (programsko opremo) sem v prihodnje napovedala v deležu 13\% poslovnih prihodkov, kar je zopet povprečje preteklih 5 let.\\

\begin{itemize}
\item{Napoved investicij v obratna sredstva}\\
Investiranje v obratna sredstva nam najbolje prikaže vrednost neto obratnega kapitala, ki ga izračunamo tako, da seštejemo zaloge, terjatve do kupcev, ostala krat. sredstva ter denar in odštejemo obveznosti do dobaviteljev ter ostale krat. obveznosti. Vse omenjene komponente sem za prihodnost napovedovala na podlagi deleža znotraj poslovnih prihodkov. V prihodnosti lahko zaznamo dva večja upada naložb v obratni kapital, ki sta posledica manjših napovedanih vrednosti kratkoročnih poslovnih terjatev in pa zalog. Za leto 2018 sem oba deleža vzela še iz prejšnjega leta, zaradi pričakovanega rahlega upada prodaje glede na leti 2016 in 2017, pa sem v prihodnje deleža zalog in terjatev v prihodkih od prodaje zmanjšala za 1 do 2\%. Poleg tega podjetje uvaja politiko vitke proizvodnje, zaradi česar se v prihodnosti pričakuje manjša stopnja kopičenja zalog.\\
Vrednost obratnega kapitala je sicer v splošnem dobro optimizirati na čim nižjo raven, saj večja vrednost pomeni zadrževanje in neizkoriščenost nekaterih sredstev podjetja. To privede do večjih stroškov financiranja, saj povečan obratni kapital pomeni, da se nekatera kratkoročna sredstva financirajo z dolgoročnimi viri, ki pa so dražji od kratkoročnih.
\end{itemize}


\subsubsection{Napoved amortizacije}:\\

Amortizacija je v zadnjem triletju proučevanih petih let predstavljala slabih 5\% poslovnih prihodkov. Po izračunih nove amortizacije iz napovedanih investicij, sem po prištevanju amortizacije obstoječih sredstev v prihodnosti prišla do manjšega deleža znotraj prihodkov od prodaje. To je pričakovano, glede na zmanjšan obseg investiranja glede na zadnje petletje. Amortizacija se tako giblje v prihodnje od 4 do 3,5\%. Amortizacijo sredstev sem računala na osnovi njihovih nabavnih vrednosti, in sicer po stopnjah, ki jih je podjetje uporabljalo do sedaj. Največje vrednosti (od 66\% do 61\%) kot pričakovano prinaša amortizacija proizvajalnih strojev in naprav. \\



\subsubsection{Vrednost prostega denarnega toka (FCF)}:\\

Za izračun prostega denarnega toka podjetja bomo izmed analiziranih računovodskih postavk potrebovali štiri ključne elemente. To so dobiček iz poslovanja po davkih, amortizacija, naložbe v osnovna sredstva in naložbe v obratni kapital. \\
Po enačbi opisani zgoraj moramo dobičku iz poslovanja torej amortizacijo prišteti, saj je to nedenarni odhodek in je dejansko na razpolago za lastnike podjetja, torej je del prostega denarnega toka. Vsoti sedaj odštevamo naložbe v osnovna sredstva, ki za podjetje ne predstavljajo prostega denarnega toke, na koncu pa odštejemo še naložbe v obratni kapital. Tako dobimo dejansko vrednost denarnega toka, ki je na razpolago lastnikom podjetja in ki jo bomo uporabili pri ocenjevanju vrednosti delnice podjetja.\\
%%% prikaz izračuna fcff
V skladu z opisanimi napovedmi sem za prihodnje petletje napovedala nekoliko višje vrednosti prostega denarnega toka, kot smo to lahko videli v preteklih 5 letih, kar pa je predvsem posledica znižanih napovedanih investicij ob hkratnih še vedno visokih prihodkih.


\subsection{OCENA DISKONTNE STOPNJE}:\\

Za pridobitev osnovne vrednosti delnice podjetja X, moram skladno s pravili metode diskontiranih denarnih tokov, določiti še diskontno stopnjo, po kateri bom diskontirala zgoraj opisane dobljene proste denarne tokove. Osnovna vrednost, ki jo bom dobila, torej velja kot tržna vrednost, kjer lastnost zaprtosti in pomanjkanja tržljivosti podjetja še ni upoštevana. Ta učinek bom dodala šele na koncu, v obliki prej omenjenih odbitkov. \\
Diskontna stopnja je pri vrednotenju podjetja ključnega pomena, saj odraža stroške kapitala. V primeru, da je višja, bo vrednost podjetja manjša, kar pomeni, da je naložba v podjetje bolj tvegana in vlagatelji za prevzem večjega tveganja pričakujejo tudi višje donose. Ker pa je običajno tvegana naožba manj priljubljena, vlagatelji raje investirajo z manjšim tveganjem, četudi ob enakih donosih.\\
Kot omenjeno, diskontna stopnja odraža strošek kapitala, pri čemer je vključen tako dolžniški kot tudi lastniški kapital. Dolžniški kapital je praviloma nižji od lastniškega, odraža pa stopnjo obrestne mere za finančne vire podjetja. V ta namen moramo pridobiti nominalno obrestno mero na posojilo, ki je izračunana kot razmerje med plačanimi obrestmi in celotnim dolgom v določenem letu. Za opazovano sem vzela leto 
2017, ko je nominalna stopnja znašala 1,82\%.  Za izračun stroška dolžniškega kapitala potrebujemo še efektivno davčno stopnjo, ki je določena kot razmerje med plačanimi davki in dohodkom pred obdavčitvijo. Pove nam, kako davčni sistem vpliva na spodbude za investicije podjetij v različne oblike naložb in kako to dalje vpliva na premoženje varčevalcev. Za izračun sem vzela dolgoročno zakonsko določeno davčno stopnjo za Slovenijo, ki znaša 19\% in preračunala višino stroška dolžniškega kapitala, 1,48\%.\\
Pri določanju stroška lastniškega kapitala sem upoštevala netvegano stopnjo donosa v višini 3,5\%, vzeto s strani Duff and Phelps in pa iz istega vira dobljeno 5- odstotno tržno premijo za tveganje, objavljeno 2.9.2017. Koeficient beta z upoštevanjem dolgov znotraj kapitala znaša 1,35. Upoštevala sem tudi odbitek za majhnost podjetja glede na tržno kapitalizacijo podjetja X, ki je objavljen na strani Duff and Phelps, in sicer v višini 3,67\%. Zadnji je še odbitek za deželno tveganje, vzet s spletne strani Damodaran, ki znaša 1,57\%. Iz opisanih vrednosti izračunana vrednost stroška lastniškega kapitala tako znaša 15,5\%.\\
Pri končnem izračunu WACC moramo upoštevati tudi razmerje med dolžniškim in lastniškim kapitalom, torej dejansko sestavo financiranja podjetja. Po zgoraj omenjeni formuli tako seštevamo utežena deleža stroška lastniškega in dolžniškega kapitala, z upoštevanjem efektivne davčne stopnje in dobimo WACC v vrednosti 11,71\%.\\
  



\subsection{IZRAČUN VREDNOSTI DELNICE PO METODI DENARNIH TOKOV}:\\

Po diskontiranju vseh izračunanih prostih denarnih tokov iz napovedi za prihodnje petletje, moramo vsoti dodati še diskontiran znesek preostale vrednosti. Predpostavljamo namreč delovanje podjetja v teoretično neskončnost, zato bodo denarne tokove ustvarjali še tudi kasneje, po letu 2022. Preostala vrednost predstavlja ravno vsoto vseh v nadaljni prihodnosti predvideno ustvarjenih prostih denarnih tokov. Predpostavila sem stopnjo rasti FCFF 1,8\%, kar je po napovedih UMAR pričakovana rast BDP v evrskem območju. Ustvarjanje prostega denarnega toka podjetja X naj bi torej v prihodnosti raslo skladno s tem deležem. Za osnovo sem po formuli za izračun preostale vrednosti vzela napovedan prost denarni tok v letu 2022 ter ga projecirala v prihodnost. Pričakovana preostala vrednost podjetja znaša tako 113,538 mio eur, ko jo diskontiramo z WACC in upoštevanjem petega zaporednega leta $((1+WACC)^5)$, pa dobimo diskontiranih dobrih 65 milijonov evrov. Vsota vrega razpoložljivega kapitala, na dan vrednotenja, znaša tako 89,086 mio evrov. Temu znesku moramo odšteti še neto dolg, saj gledamo le vrednost lastniškega kapitala in tako dobimo 72,798 mio evrov. Da bi dobili znesek, ki odpade na eno delnico, moramo to vrednost deliti s številom delnic, ki skozi vsa proučevana leta znaša 514215. Dobimo tako tržno vrednost ene delnice, in sicer 141,6 evrov. \\
Kot že omejeno, ta vrednost ne odraža realne cene delnice, saj je podjetje lastniško zaprto in predpostavke o popolni tržljivosti tu ne smemo uporabiti. Za ta namen bom v sledečem poglavju na podlagi odbitkov določila pravo vrednost delnice.\\

\subsubsection{Določanje odbitkov izračunani tržni vrednosti}:\\

\begin{itemize}
\item{\textbf{Odbitek zaradi pomanjkanja obvladljivosti}}

Ker smo z opisanim izračunom pridobili tržno vrednost podjetja za manjšinskega lastnika (vrednost ene delnice), moramo najprej na to vrednost aplicirati vpliv manjše možnosti obvladovanja podjetja z vidika lastnika ene delnice.\par
Lastnik delnice podjetja X ima navzgor omejeno pravico glasovanja na skupščinah, in sicer v vrednosti 2\% vseh izdanih glasovalnih delnic, ne glede na število delnic, ki jih ima v lasti. Takšno pravilo omejuje v enaki meri tako manjšinskega kot tudi večinskega lastnika in ju poenoti, zato bi bil obravnavani odbitek manjši. Poleg tovrstne enakopravnosti imajo tako manjšinski kot večinski lastniki v podjetju X zaradi internega lastništva večji dostop do informacij o poslovanju in so manj podvrženi tveganju neinformiranja in netransparentnosti. Cilj podjetja je namreč združiti delničarje in ohraniti delništvo znotraj podjetja, kar je manjšinskemu lastniku v prid. Delničarji, ki lahko enostavneje dostopajo do pomembnih informacij, imajo poleg tega še posebne omejitve kdaj lahko svoje delnice prodajajo ali kupujejo, o svojem namenu pa morajo obvestiti tudi na skupščini, tako da je informiranje celotne skupine delničarjev še bolj zaščiteno in zagotovljeno. \\
Izračun odbitka za pomanjkanje obvladljivosti izhaja iz podatka o kontrolni premiji. To je pribitek, plačan za prevzem podjetja, v primerjavi s cenami delnic tega podjetja na organiziranem trgu kapitala, pred objavo prevzema. (dipl. Bojan Praznik) Tak pribitek je plačan z namenom pridobivanja večjega deleža lastništva in s tem kontrole v podjetju.\\
Za določitev kontrolne premije sem se oprla na več študij. Prva je študija angleškega podjetja RSM, ki se ukvarja med drugim tudi s poslovnim svetovanjem in svetovanjem upravljanja tveganja. Raziskovali so višino povprečne kontrolne premije v letu 2017 glede na več dejavnikov, med katerimi pa sem sama za najbolj relevantnega izbrala industrijsko dejavnost, s katero se podjetje ukvarja. V skladu s tem torej podjetju X pripada kontrolna premija v višini 30,3\%. Ker je bila študija izpeljana v okviru podatkov pomembnejših Avstralskih podjetij, sem za primerjavo poiskala še raziskavo VZMD, vseslovenskega združenja malih delničarjev. Ocenjujejo, da povprečna prevzemna premija na Slovenskem prostoru znaša okoli 25\%, prav zaradi manjše likvidnosti in razvitosti trga. Hkrati ugotavljajo, da so pretekli prevzemi potekali z najvišjimi stopnjami ravno na področju predelovanih dejavnosti, na bolj razvitih trgih pa naj bi se delež gibal nekje okoli 40\%. Tretja preučevana raziskava pa je študija Mergerstat, Shannon Pratt' Control Premium Study. Leta 2003 je bila izvedena raziskava med nakupi podjetij v letih 1986 in 2002. Preučevali so povprečne kontrolne premije manjšinskih in večinskih delničarjev. Ker pa mi ocenjujemo vrednost delnice manjšinskega lastnika, sem upoštevala povprečno kontrolno premijo, ki so jo bili pripravljeni plačati mali delničarji. Ta znaša približno 35\%. Druga študija istega vira deli kontrolne premije tudi po industrijah. Pri tem predelovalni industriji pripisuje srednjo kontrolno premijo 31\%, še bolj podrobno na področju proizvodnje električnih naprav pa so izračunali mediano kontrolnih premij pri 33,4\%.\\
Za relevantno stopnjo kontrolne premije sem zato vzela vrednost 33,4\% in jo upoštevala (po formuli navedeni zgoraj) v izračunu odbitka za neobvladljivost, ki tako znaša 25,04\%.\\





\item{\textbf{Odbitek zaradi pomanjkanja tržljivosti}}

Pomanjkanje tržljivosti naložbe v delnice podjetja X je povzročitelj drugega, za nas najpomembnejšega odbitka.\par
Iz opisanih značilnosti delnic podjetja lahko vidimo, da ima njegov delničar že statuarno precej omejitev pri prodaji svoje naložbe, zato bo odbitek predvidoma višji. Poleg tega je zaradi zaprtosti podjetja manj potencialnih kupcev, kar prodajo še dodatno otežuje, verjetnost kotiranja na borzi pa je skladno s cilji o ohranjanju delništva znotraj podjetja praktično ničelna. Na drugi strani podjetje delničarjem redno izplačuje dividende, posluje transparentno in seznanja delničarje in javnost z informacijami o poslovanju, njegovo dosedanje poslovanje pa je naklonjeno rasti in razvoju. Vse to vpliva na zmanjšanje odbitka zaradi manjše tržljivosti.\\

Za bolj natančno določitev odbitka za pomanjkanje tržljivosti, ki vpliva na vrednost delnice pri primeru podjetja X, sem se oprla na raziskave \underline{Houlihan Lokey Howard and Zukin Investment Banking Services}. Osredotočali so se na že izvedene empirične študije (26\% mediana odbitkov), na rezultate sodnih primerov (25\% mediana) in na odbitke pri transakcijah na trgu nepremičnin.\\
V sklopu empiričnih študij so vzeli 16 raziskav, izvedenih v letih 1970 do 2003, od katerih 2 temeljita na vrednosti delnic pred prvo javno ponudbo, ostalih 14 pa se nanaša na vrednosti restriktivnih delnic podjetij. V tabeli sem na kratko povzela rezultate vsake od raziskav in njene glavne značilnosti in ugotovitve. Ker je na podlagi restriktivnih delnic izpeljanih več raziskav, poleg tega pa so strokovnjaki tej vrsti študij bolj naklonjeni, sem pri ocenjevanju upoštevala predvsem omenjenih 14 raziskav. Kljub temu pa ima naše podjetje nekatere specifične lastnosti, ki niso enakovredno zajete v vseh raziskavah, zato ne moremo vseh obravnavati enako.\par

Podjetje X je glede na svojo konkurenco dobro stoječe, njegov trg je velik in še raste, izdelujejo nove produkte in širijo svojo ponudbo. Poslovanje v zadnjih 5 letih je spodbudno za prihodnost, takšne pa so tudi moje napovedi. Poleg tega je podjetje v preteklosti redno izplačevalo dividende, kar predpostavljamo tudi v prihodnje, tveganost dolge povratne dobe naložbe v delnico je zato majhna. Podjetje posluje transparentno in seznanja delničarje in javnost z informacijami o poslovanju. Ti imajo posledično večji nadzor nad gibanjem vrednosti naložbe in je želja po vključitvi med lastnike večja. Vse to pa torej vpliva na zmanjšanje odbitka zaradi manjše tržljivosti. Po drugi strani je verjetnost kotiranja podjetja na borzi skladno s cilji o ohranjanju delništva znotraj podjetja praktično ničelna, kar pomeni, da bo število potencialnih kupcev ostalo približno enako, povečevalo se bo le z dodatnim zaposlovanjem. To pomeni težjo prodajljivost delnice in posledično večji odbitek za pomanjkanje tržljivosti.\\
V sklopu obravnavanih raziskav je ena najmlajših raziskava \underline{FMV Opinions}, v kateri je upoštevana delitev odbitkov glede na SIC kode dejavnosti. Našemu podjetju raziskava pripisuje v povprečju 23,3\% odbitek. Točno enako vrednost dobimo po rezultatih sicer iste raziskave, ko so namesto SIC kode dejavnosti za kriterij vzeli način prodaje delnic (na borzi ali izven nje). Druga pomembnejša je še študija \underline{Management planning}, ki zajema transakcije pri 115 podjetjih, ki so jih glede na prihodke od prodaje, dobiček, tržno ceno delnice, njeno stabilnost ipd. razdelili na 4 kvantile in vsakemu določili povprečni odbitek pri vsakem od parametrov. Našemu podjetju po tej raziskavi pripada odbitek 25,4\%. Malo starejša, a zelo merodajna je tudi raziskava Institutional investor study, ki upošteva prodajo, zaslužek, način prodaje delnic,... Ta podjetju X pripisuje enkrat med 10 in 20 odstotni odbitek, drugič pa od 30 do 40 odstotnega. Vse ostale omenjene študije so izpeljane na med seboj zelo podobnem principu, proučujejo pa predvsem investicijske sklade, ki vlagajo v podjetja katerih restriktivne delnice raziskujejo. Njihovi zaključni odbitki se v povprečju gibljejo malo nad 30\%.\par
Sama sem glede na vse opisane rezultate in glede na zgoraj navedene lastnosti delništva v podjetju X, pri vrednotenju upoštevala odbitek v višini 25\%.\\


\subsubsection{Končna vrednost delnice z upoštevanimi odbitki}:\\

Imamo torej vrednosti obeh pomembnejših odbitkov, ki jih sedaj lahko apliciramo na prvotno dobljeno vrednost delnice. Po formuli 

\begin{equation}
vrednost\ po\ odbitkih=vrednost\ pred\ odbitki *{(1-odbitek_1)}*{(1-odbitek_2)}
\end{equation}
\\
kjer za $odbitek_1$ vzamemo odbitek za neobvladljivost in za $odbitek_2$ za pomanjkanje tržljivosti, dobimo za končno vrednost delnice ceno 79,5 evrov.



\end{itemize}



% slovar
\section*{Slovar strokovnih izrazov}

%\geslo{}{}
%
%\geslo{}{}
%


% seznam uporabljene literature
\begin{thebibliography}{99}



%\bibitem{}

\end{thebibliography}

\end{document}










