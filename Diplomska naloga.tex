\documentclass{article}
\usepackage{graphicx}

\begin{document}

\title{Diplomska naloga:\\
\textbf{Posebnosti pri vrednotenju podjetja z internim trgom delnic - Domel Holding d.d.} }

\author{Avtorica: Neža Habjan\\
Mentor: doc. dr. Matjaž Črnigoj}
\date{\today}

\maketitle
\newpage

\section{Ocenjevanje vrednosti}
\subsection{Metode ocenjevanja vrednosti}
Pri ocenjevanju vrednosti poznamo tri glavne načine vrednotenja, od katerih vsi temeljijo na ekonomskih načelih ravnovesja cen, pričakovanih koristi ali substituciji. To so način tržnih primerjav, nabavnovrednostni in na donosu zasnovan način, vsak od njih pa vključuje podrobnejše metode uporabe. Med temi moramo izbirati previdno in v obzir vzeti vrsto sredstva, ki ga ocenjujemo, razpoložljive informacije, ki jih imamo, naš namen ocenjevanja ter ustrezne prednosti in slabosti posamezne metode. 

\subsubsection{Način tržnih primerjav}
Vrednost ocenjevanega sredstva pri tem načinu dobimo preko primerjave z vrednostmi podobnih sredstev, ki so že ovrednotena, torej so informacije o cenah že na voljo. Uporaba pride v poštev predvsem v primerih, ko je bilo ocenjevano sredstvo nedavno prodano v poslu, ki je primeren za proučevanje, ko se sredstvo dejavno javno trži, ali pa s podobnimi sredstvi obstajajo nedavni posli. Kadar se primerljiva tržna informacija ne nanaša na točno ali v bistvu enako sredstvo, mora ocenjevalec vrednosti opraviti primerjalno analizo tako podobnosti kot tudi razlik po kakovosti in količini med primerljivimi sredstvi in ocenjevanim sredstvom. 
Metode načina tržnih primerjav so :
\begin{itemize}
\item metoda primerljivih poslov
\item metoda primerljivih podjetij, uvrščenih na borzi
\end{itemize}
Ker pa pri primerjavi z enakimi ali podobnimi sredstvi, ki so že ovrednotena, moramo upoštevati prilagoditve razlik od ocenjevanega sredstva, se pri tej metodi vračunajo tudi razni odbitki (npr. za tržljivost) in pribitki (npr. za obvladovanje), katerih podrobnejši opis sledi v nadaljevanju naloge.

\subsubsection{Nabavnovrednostni način}
Ta način podaja vrednost sredstva z izračunavanjem sedanje nadomestitvene ali nabavne vrednosti. Gre torej za ceno pridobitve sredstva enake koristnosti z nakupom ali z gradnjo oziroma izdelavo. Cena naj bi zajemala vse stroške, ki bi jih imel značilni udeleženec trga. Način naj bi se uporabljal v okoliščinah, ko bi tržni udeleženci lahko ponovno ustvarili sredstvo skoraj enake koristnosti, kot je ocenjevano sredstvo, brez regulativnih ali zakonskih omejitev ter v primerih, ko se že za izračun podlage vrednosti uporablja nadomestitvene stroške npr. nadomestitveno nabavno vrednost. Pri tem z izrazom podlaga vrednosti mislimo bodisi na tržno, pošteno vrednost, vrednost za naložbenika,... glede na to, kaj si izberemo za ocenjevanje. Poleg tega ima nabavnovrednostni način veliko težo v primerih, ko sredstvo neposredno ne ustvarja dohodka in zaradi njegove enkratne vrste noben od ostalih dveh načinov ne pride v poštev. 
Metode nabavnovrednostnega načina so :
\begin{itemize}
\item metoda nadomestitvene vrednosti
\item metoda reprodukcijske nabavne vrednosti
\item metoda seštevanja
\end{itemize}
Pri uporabi metod opisanega načina moramo upoštevati tudi odbitke za fizično poslabšanje in vse druge pomembne oblike zastarelosti.

\subsubsection{Na donosu zasnovan način}
Vrednost sredstva dobimo s pretvorbo prihodnjih denarnih tokov, ki nam jih sredstvo prinaša, v eno samo sedanjo vrednost. Sklicujemo se torej na vrednost dohodka, denarnih tokov ali prihranka, ki ga sredstvo ustvari v prihodnosti. Način je najbolj uporaben v primerih, ko je ustvarjanje denarnih tokov bistvena sestavina, ki vpliva na stališče tržnega udeleženca o vrednosti sredstva. Hkrati pa je dobra podlaga za ocenjevanje tudi kadar imamo o sredstvu veliko podatkov o njegovih prihodnjih donosih, trenutnih tržnih primerljivk pa je malo ali jih sploh ni. Temeljna podlaga na donosu zasnovanega načina je, da naložbeniki pričakujejo donos od svojih naložb in da naj bi tak donos odražal zaznano raven tveganja pri naložbi. V ta namen moramo določiti ustrezno diskontno stopnjo, s katero bomo prihodnje vrednosti prilagodili na sedanjost.
Metoda na donosu zasnovanega načina je :
\begin{itemize}
\item metoda diskontiranega denarnega toka
\end{itemize}
Pri tej metodi poznamo različne variacije, saj lahko denarne tokove diskontiramo na različne načine. Izbiramo namreč vrsto denarnega toka, dolžino obdobja podrobne napovedi, način napovedi denarnega toka, načina izračuna preostale vrednosti, diskontno vrednost, ... 

\subsection{Standardi sredstev}
Standardi se lahko nanašajo na različne vrste sredstev, kot na primer podjetja in poslovne deleže, neopredmetena sredstva, naprave in opremo, pravice na nepremičninah, nepremičnine za gradnjo, finančne instrumente,... V diplomski nalogi obravnavam ocenjevanje podjetja, zato se osredotočimo na prvo vrsto sredstev, torej podjetja in poslovne deleže.
\subsubsection{Podjetja in poslovni deleži}

\end{document}
