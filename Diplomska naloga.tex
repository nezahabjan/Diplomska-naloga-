\documentclass[12pt,a4paper]{amsart}
% ukazi za delo s slovenscino -- izberi kodiranje, ki ti ustreza
\usepackage[slovene]{babel}
%\usepackage[cp1250]{inputenc}
%\usepackage[T1]{fontenc}
\usepackage[utf8]{inputenc}
\usepackage{amsmath,amssymb,amsfonts}
\usepackage{url}
%\usepackage[normalem]{ulem}
\usepackage[dvipsnames,usenames]{color}

% ne spreminjaj podatkov, ki vplivajo na obliko strani
\textwidth 15cm
\textheight 24cm
\oddsidemargin.5cm
\evensidemargin.5cm
\topmargin-5mm
\addtolength{\footskip}{10pt}
\pagestyle{plain}
\overfullrule=15pt % oznaci predolgo vrstico


% ukazi za matematicna okolja
\theoremstyle{definition} % tekst napisan pokoncno
\newtheorem{definicija}{Definicija}[section]
\newtheorem{primer}[definicija]{Primer}
\newtheorem{opomba}[definicija]{Opomba}

\renewcommand\endprimer{\hfill$\diamondsuit$}


\theoremstyle{plain} % tekst napisan posevno
\newtheorem{lema}[definicija]{Lema}
\newtheorem{izrek}[definicija]{Izrek}
\newtheorem{trditev}[definicija]{Trditev}
\newtheorem{posledica}[definicija]{Posledica}


% za stevilske mnozice uporabi naslednje simbole
\newcommand{\R}{\mathbb R}
\newcommand{\N}{\mathbb N}
\newcommand{\Z}{\mathbb Z}
\newcommand{\C}{\mathbb C}
\newcommand{\Q}{\mathbb Q}


% ukaz za slovarsko geslo
\newlength{\odstavek}
\setlength{\odstavek}{\parindent}
\newcommand{\geslo}[2]{\noindent\textbf{#1}\hspace*{3mm}\hangindent=\parindent\hangafter=1 #2}


% naslednje ukaze ustrezno popravi
\newcommand{\program}{Finančna matematika} % ime studijskega programa: Matematika/Finan"cna matematika
\newcommand{\imeavtorja}{Neža Habjan} % ime avtorja
\newcommand{\imementorja}{prof.~doc.~dr. Matjaž Črnigoj} % akademski naziv in ime mentorja
\newcommand{\naslovdela}{Posebnosti pri vrednotenju podjetja z internim trgom delnic - Domel Holding d.d.}
\newcommand{\letnica}{2019} %letnica diplome


% vstavi svoje definicije ...




\begin{document}

% od tod do povzetka ne spreminjaj nicesar
\thispagestyle{empty}
\noindent{\large
UNIVERZA V LJUBLJANI\\[1mm]
FAKULTETA ZA MATEMATIKO IN FIZIKO\\[5mm]
\program\ -- 1.~stopnja}
\vfill

\begin{center}{\large
\imeavtorja\\[2mm]
{\bf \naslovdela}\\[10mm]
Delo diplomskega seminarja\\[1cm]
Mentor: \imementorja}
\end{center}
\vfill

\noindent{\large
Ljubljana, \letnica}
\pagebreak

\thispagestyle{empty}
\tableofcontents
\pagebreak

\thispagestyle{empty}
\begin{center}
{\bf \naslovdela}\\[3mm]
{\sc Povzetek}
\end{center}
% tekst povzetka v slovenscini
V povzetku na kratko opi"si vsebinske rezultate dela. Sem ne sodi razlaga organizacije dela -- v katerem poglavju/razdelku je kaj, pa"c pa le opis vsebine.
\vfill
\begin{center}
{\bf Angle"ski naslov dela}\\[3mm] % prevod slovenskega naslova dela
{\sc Abstract}
\end{center}
% tekst povzetka v anglescini
Prevod zgornjega povzetka v angle"s"cino.

\vfill\noindent
{\bf Math. Subj. Class. (2010):} navedi vsaj eno klasifikacijsko oznako -- dostopne so na \url{www.ams.org/mathscinet/msc/msc2010.html}  \\[1mm]
{\bf Klju"cne besede:} navedi nekaj klju"cnih pojmov, ki nastopajo v delu  \\[1mm]
{\bf Keywords:} angle"ski prevod klju"cnih besed
\pagebreak



% tu se zacne besedilo seminarja
\section{Uvod}
Podjetja po svetu dandanes delujejo v mnogih različnih oblikah, znotraj teh pa se ustvarjajo razlike še glede na organiziranost, vodenje, smernice delovanja,... Vse to ustvarja veliko heterogenost v podjetniškem prostoru. Seveda pa je, če želimo, da podjetje posluje uspešno in z dobičkom, potrebno vse te dejavnike izbirati skrbno in premišljeno. Vsaka vrsta organiziranosti namreč ne ustreza vsem trgom, delavcem, panogam,... \\
V slovenskem prostoru se tako nahaja podjetje, ki je po svoji organiziranosti unikat. Kljub temu, da je precej zaprto, kar se tiče vlaganja tretjih oseb v njegove delnice, posluje z odliko in si v Evropi lasti 60\% tržni delež na področju sesalnih enot, ki so njegov glavni proizvod. To je podjetje Domel, s sedežem v Železnikih, v Selški dolini. Domel je delniška družba, ki s svojimi delnicami upravlja na zaprtem, internem trgu. Neizpostavljenost širšemu trgu ali borzi zato povzroča precej nizko vrednost le teh, še posebej, ker podjetje uživa izjemno dobro ime v javnosti, kar bi za vlagatelje lahko pomenilo velik interes.\\
V svojem diplomskem delu bi zato rada analizirala in ocenila padec vrednosti delnice Domela, zaradi njegove zaprtosti. Pri tem bom preučila obnašanje delničarjev podjetja, torej če in kdaj imajo interes povečati oziroma zmanjšati svojo naložbo vanj. Ob vrednotenju bom kot glavnega morala oceniti diskontni faktor, s katerim bom prihodnje denarne tokove prenesla na sedanjo vrednost. Pri tem si bom pomagala z dvema pomembnejšema konceptoma odbitkov, ki sta zaradi zaprtosti podjetja ključnega pomena, in sicer s tržljivostjo in likvidnostjo.
\newpage

\section{Predstavitev podjetja Domel Holding d.d.}
Domel je eno večjih slovenskih industrijskih podjetij, ki izdeluje električne motorje in komponente iz laminatov, aluminija, termo plastike,... Njihovi izdelki se uporabljajo večinoma za vgradnjo v vakuumske enote, pa tudi na področju bele tehnike, prezračevanja, avtomobilske proizvodnje, medicine,...
Podjetje je bilo ustanovljeno leta 1946 in je v rekordnem letu 2006 izdelalo skoraj 6,5 milijona različnih elektromotorjev in ustvarilo promet v višini 81,9 milijona evrov. Največji izvozni trg jim predstavlja Nemčija, v manjši meri pa sodelujejo tudi z Madžarsko, Švedsko, Poljsko, Italijo, Avstrijo, Romunijo,... Zgovoren podatek o njihovi uspešnosti je tudi dejstvo, da je, kot že omenjeno zgoraj, v Evropi v kar šestih od desetih prodanih sesalnikov vgrajen motor, proizveden v podjetju Domel. V svetovnem merilu sodelujejo s podjetji kot so Elektrolux, Philips, Rowenta, Stihl, Husqvarna, Samsung,... dve svoji proizvodnji enoti pa imajo poleg glavne v Železnikih tudi v Retečah blizu Škofje Loke in na Kitajskem.

\subsection{Kratek pregled zgodovine podjetja}
Podjetje Domel ima zelo pestro zgodovino, tako z vidika menjave področij delovanja, kakor tudi načina in organizacije vodstva. Prvo obdobje, od ustanovitve leta 1946 pa do leta 1962 je bilo podjetje poznano pod imenom NIKO, ukvarjalo pa se je predvsem s predelavo in izdelavo kovinskih izdelkov.Omenjenega leta 1962 pa se je pridružilo Iskri, katere ime nosi vse do leta 1991. Tedaj, z razpadom Jugoslavije, se za krajši čas podjetje preimenuje v Elektromotorji d.o.o., šele leto kasneje pa postane Domel, in sicer od 1994 Domel d.d. V letih po tem podjetje odpre svojo proizvodnjo enoto na Kitajskem, s čimer se le še trdneje uveljavi v svetovnem merilu. V letu 2010 pa večjo prelomnico za podjetje pomeni preoblikovanje iz delniške družbe v družbo z omejeno odgovornostjo, ki še vedno ostaja v lasti zaposlenih, bivših zaposlenih in upokojencev, kar pa je v slovenskem gospodarskem prostoru edinstven primer.

\subsubsection{Družba pooblaščenka}
Leta 1998, v času množičnega lastninjenja slovenskih podjetij, so delavci podjetja Domel reorganizirali združenje notranjih delničarjev v Družbo pooblaščenko Domel d.d. To so storili iz strahu pred domnevnim sovražnim prevzemom podjetja s strani ameriške družbe Ametek, ki naj bi Domel želel prevzeti zaradi velikega tržnega deleža v Evropi (20\%). Prihodnost pooblaščenke je bila sicer zelo negotova in mnogi so obetali propad podjetja, v primeru da ta ne dobi močnega strateškega partnerja. Nejasno prihodnost združevanja so prikazovali v zvezi z nejasno situacijo delnic v pooblaščenki, z nezmožnostjo samostojnega preživetja matičnega podjetja, z osebnimi interesi tistih, ki vodijo projekt ustanavljanja pooblaščenke, itd. Ampak strah delavcev je bil močnejši. Družbo so ustanovili postopoma, in sicer je skupina glavnih zastopnikov delničarjev pričela odkupovati delnice vseh ostalih in tako večati svoj lastniški delež. Posledično so na odločilni skupščini ohranili vlogo najmočnejšega lastnika matičnega podjetja in s tem zavarovali lastne interese ter preprečili sovražni prevzem. Delnice je pooblaščenka v veliki meri predala matičnemu podjetju, ki jih je nato razdelilo svojim zaposlenim. Tako danes lastniki podjetja v 92,96\% deležu ostajajo zaposleni, bivši zaposleni in upokojenci. Ostalih 7,04\% pa ostaja delnic vplačanih z denarjem, torej ne nujno del pooblaščenke. Tudi te z dohodkom iz dejavnosti, ki jih opravlja, pooblaščenka odkupuje in tako še veča delež lastništva matičnega podjetja. \\
Ustanovitev družbe pooblaščenke je za Domel pomenilo izboljšanje poslovanja in uspešno izpolnjevanje zastavljenih ciljev. Proizvodnja motorjev, ki so Domelov glavni program, se je povečala za 50\%, močno pa se je povečala tudi prodaja v ZDA. Dobiček se je povečal za 4 krat, investicije v novo tehnologijo pa za 4,5 krat. V tem času so ohranili prav vse kupce in pridobili nove na tržiščih, na katerih Domel še ni bil prisoten.\\
Po drugi strani pa opisana reorganizacija ni prinesla le pozitivnih posledic. Ker je pooblaščenka izrazito zaprta družba, njene delnice niso izpostavljene tržišču in imajo posledično nižjo vrednost, kot bi jo imele sicer. Do neke mere zato manjko pri kapitalskem donosu delnice kompenzirajo z nekoliko višjimi dividendami.



\subsubsection{Domel Holding d.d. - delniška družba}
Kot razloženo je torej Domel Holding d.d. delniška družba, ki danes ostaja edini lastnik podjetja Domel. 




\section{Ocenjevanje vrednosti}
\subsection{Metode ocenjevanja vrednosti}
Pri ocenjevanju vrednosti poznamo tri glavne načine vrednotenja, od katerih vsi temeljijo na ekonomskih načelih ravnovesja cen, pričakovanih koristi ali substituciji. To so način tržnih primerjav, nabavnovrednostni in na donosu zasnovan način, vsak od njih pa vključuje podrobnejše metode uporabe. Med temi moramo izbirati previdno in v obzir vzeti vrsto sredstva, ki ga ocenjujemo, razpoložljive informacije, ki jih imamo, naš namen ocenjevanja ter ustrezne prednosti in slabosti posamezne metode. 

\subsubsection{Način tržnih primerjav}
Vrednost ocenjevanega sredstva pri tem načinu dobimo preko primerjave z vrednostmi podobnih sredstev, ki so že ovrednotena, torej so informacije o cenah že na voljo. Uporaba pride v poštev predvsem v primerih, ko je bilo ocenjevano sredstvo nedavno prodano v poslu, ki je primeren za proučevanje, ko se sredstvo dejavno javno trži, ali pa s podobnimi sredstvi obstajajo nedavni posli. Kadar se primerljiva tržna informacija ne nanaša na točno ali v bistvu enako sredstvo, mora ocenjevalec vrednosti opraviti primerjalno analizo tako podobnosti kot tudi razlik po kakovosti in količini med primerljivimi sredstvi in ocenjevanim sredstvom. 
Metode načina tržnih primerjav so :
\begin{itemize}
\item metoda primerljivih poslov
\item metoda primerljivih podjetij, uvrščenih na borzi
\end{itemize}
Ker pa pri primerjavi z enakimi ali podobnimi sredstvi, ki so že ovrednotena, moramo upoštevati prilagoditve razlik od ocenjevanega sredstva, se pri tej metodi vračunajo tudi razni odbitki (npr. za tržljivost) in pribitki (npr. za obvladovanje), katerih podrobnejši opis sledi v nadaljevanju naloge.\\
Način tržnih primerjav zahteva obstoj podobnega sredstva ocenjevanemu, ki je že bilo ovrednoteno. Ker pa je podjetje Domel Holding d.d., kot bomo videli kasneje, unikatno predvsem zaradi zaprtosti svojega trga delnic in hkratni izjemni uspešnosti pri poslovanju, za vrednotenje opisan način ni primeren. 

\subsubsection{Nabavnovrednostni način}
Ta način podaja vrednost sredstva z izračunavanjem sedanje nadomestitvene ali nabavne vrednosti. Gre torej za ceno pridobitve sredstva enake koristnosti z nakupom ali z gradnjo oziroma izdelavo. Cena naj bi zajemala vse stroške, ki bi jih imel značilni udeleženec trga. Način naj bi se uporabljal v okoliščinah, ko bi tržni udeleženci lahko ponovno ustvarili sredstvo skoraj enake koristnosti, kot je ocenjevano sredstvo, brez regulativnih ali zakonskih omejitev ter v primerih, ko se že za izračun podlage vrednosti uporablja nadomestitvene stroške npr. nadomestitveno nabavno vrednost. Pri tem z izrazom podlaga vrednosti mislimo bodisi na tržno, pošteno vrednost, vrednost za naložbenika,... glede na to, kaj si izberemo za ocenjevanje. Poleg tega ima nabavnovrednostni način veliko težo v primerih, ko sredstvo neposredno ne ustvarja dohodka in zaradi njegove enkratne vrste noben od ostalih dveh načinov ne pride v poštev. 
Metode nabavnovrednostnega načina so :
\begin{itemize}
\item metoda nadomestitvene vrednosti
\item metoda reprodukcijske nabavne vrednosti
\item metoda seštevanja
\end{itemize}
Pri uporabi metod opisanega načina moramo upoštevati tudi odbitke za fizično poslabšanje in vse druge pomembne oblike zastarelosti.\\
Delnice podjetja Domel, ki jim ocenjujem vrednost, v prihodnosti prinašajo denarne tokove, ki jih lahko zanesljivo določimo, saj podjetje ni zagonsko ali v zgodnji fazi, pač pa že močno razvito. Prav zato je najbolj ustrezen način za vrednotenje na donosu osnovan način.

\subsubsection{Na donosu zasnovan način}
Vrednost sredstva dobimo s pretvorbo prihodnjih denarnih tokov, ki nam jih sredstvo prinaša, v eno samo sedanjo vrednost. Sklicujemo se torej na vrednost dohodka, denarnih tokov ali prihranka, ki ga sredstvo ustvari v prihodnosti. Način je najbolj uporaben v primerih, ko je ustvarjanje denarnih tokov bistvena sestavina, ki vpliva na stališče tržnega udeleženca o vrednosti sredstva. Hkrati pa je dobra podlaga za ocenjevanje tudi kadar imamo o sredstvu veliko podatkov o njegovih prihodnjih donosih, trenutnih tržnih primerljivk pa je malo ali jih sploh ni. Temeljna podlaga na donosu zasnovanega načina je, da naložbeniki pričakujejo donos od svojih naložb in da naj bi tak donos odražal zaznano raven tveganja pri naložbi. V ta namen moramo določiti ustrezno diskontno stopnjo, s katero bomo prihodnje vrednosti prilagodili na sedanjost.
Metoda na donosu zasnovanega načina je :
\begin{itemize}
\item metoda diskontiranega denarnega toka
\end{itemize}
Pri tej metodi poznamo različne variacije, saj lahko denarne tokove diskontiramo na različne načine. Izbiramo namreč vrsto denarnega toka, dolžino obdobja podrobne napovedi, način napovedi denarnega toka, načina izračuna preostale vrednosti, diskontno vrednost, ... 

\subsection{Standardi sredstev}
Standardi se lahko nanašajo na različne vrste sredstev, kot na primer podjetja in poslovne deleže, neopredmetena sredstva, naprave in opremo, pravice na nepremičninah, nepremičnine za gradnjo, finančne instrumente,... V diplomski nalogi obravnavam ocenjevanje podjetja, zato se bom osredotočila na prvo vrsto sredstev, torej podjetja in poslovne deleže.
\subsubsection{Podjetja in poslovni deleži}
Do pojma podjetje se pri vrednotenju opredelimo v odvisnosti od namena ocenjevanja njegove vrednosti. Na splošno pa podjetja opravljajo komercialne, industrijske, storitvene ali naložbene dejavnosti. Lahko se nahajajo v različnih oblikah, kot na primer kapitalske, osebne družbe, samostojni podjetniki, ... Pri ocenjevanju se bom osredotočila na oceno lastniškega deleža podjetja Domel, torej na vrednost ene delnice podjetja. Za podlago vrednosti bi lahko izbrala tudi  vrednost celotnega kapitala, vrednost vloženega kapitala, vrednost iz poslovanja,... vendar je glede na to, da je Domel delniška družba, katere posebnost je prav njen trg delnic, najbolj primeren za ocenjevanje lastniški kapital.  







% slovar
\section*{Slovar strokovnih izrazov}

%\geslo{}{}
%
%\geslo{}{}
%


% seznam uporabljene literature
\begin{thebibliography}{99}
https://sl.wikipedia.org/wiki/Domel
https://www.domel.com/sl/podjetje/predstavitev


%\bibitem{}

\end{thebibliography}

\end{document}










